\documentclass[12pt]{article}
\usepackage{amsmath}
\usepackage{fancyheadings}

\setlength{\oddsidemargin}{0.0in}
\setlength{\textwidth}{6.5in}
\setlength{\topmargin}{-0.5in}
\setlength{\textheight}{9.0in}

\bibliographystyle{unsrt}
\renewcommand{\baselinestretch}{1.0}
\newcommand{\matexpr}{\texttt{matexpr}}
\newcommand{\gOR}{ $|$ }

\title{\matexpr\ User Guide}
\author{D. Bindel}

\pagestyle{fancy}

\headrulewidth 1.5pt
\footrulewidth 1.5pt
\chead{\matexpr}
\lhead{}
\rhead{}
\cfoot{\thepage}
\begin{document}

\maketitle


\section{Introduction}

\matexpr\ is a source-to-source translator for embedding simple MATLAB-like
matrix expressions in C/C++.  \matexpr\ interprets specially-formatted
comments in a source file and uses them to generate ordinary C
code.  For example, the following code computes a Rayleigh quotient
for two three-by-three matrices:
\begin{verbatim}
double rayleigh_quotient3d(double* K, double* M, double* v)
{
    double rq;
    /* <generator matexpr>
     // Compute the Rayleigh quotient for a 3-by-3 pencil (K,M)

     output rq;
     input K(3,3), M(3,3), v(3);
     rq = (v'*K*v)/(v'*M*v);
    */
    return rq;
}
\end{verbatim}

In addition to MATLAB-like matrix construction and arithmetic, \matexpr\
also provides simple symbolic differentiation.

\matexpr\ is \emph{not} a full package for numerical linear algebra, nor
even a particularly good substitute for a decent C++ matrix class.
The purpose of \matexpr\ is to make it easy to avoid index errors and
unnecessary overhead when evaluating the sorts of small matrix expressions
that arise in coding finite elements and other similar tasks.


\section{\matexpr\ command line}

The {\tt matexpr} command line has the following form:
\begin{verbatim}
  matexpr [-comment] [-nogen] [-check] infile
\end{verbatim}
where
\begin{itemize}

\item {\tt -comment} specifies that \matexpr\ should output
  labels in generated code to specify corresponding source lines.
  This is mostly useful for debugging generated code.

\item {\tt -line} specifies that \matexpr\ should output C preprocessor
  \verb|#line| labels so that error diagnostics from the C/C++ compiler
  will point to the appopriate place in the input file.

\item {\tt -nogen} specifies that \matexpr\ should remove all
  automatically generated code from the output file.

\item {\tt -check} specifies that \matexpr\ should check the
  input file without generating any other output.

\item {\tt -c99complex} specifies that \matexpr\ should use C99-style
  complex numbers (as opposed to C++ style complex).

\end{itemize}


\section{Interface syntax}

The complete syntax for \matexpr\ is given in Figure~\ref{matexpr-syntax-fig}. 
Matrices must have known {\em constant} dimensions.  Variables that are
not explicitly declared for input or output are assumed to be scratch
variables.

\matexpr\ expressions are embedded in C-style comments that begin with
the start-of-comment string {\tt /* <generator matexpr>}.  The
starting tag can include an optional assignment of the form {\tt
  complex=''{\it name}''} to specify a type to be used locally for
complex inputs.  The generator finishes processing at the end of the C
comment.  C++-style line comments may be used to document the
generator code.  The output of the generator is also marked off by
special comments, i.e.
\begin{verbatim}
  /* <generated matexpr> */ {
  ... Generated source goes here ...
  } /* </generated> */
\end{verbatim}
The generator will skip any code in the input file which has this form.
Consequently, if {\tt foo1.cc} is a valid input file and we run
\begin{verbatim}
  matexpr foo1.cc > foo2.cc
  matexpr foo2.cc > foo3.cc
\end{verbatim}
then the files {\tt foo2.cc} and {\tt foo3.cc} will be identical.

\begin{figure}
\begin{center}
\begin{tabular}{l@{ := }l}
  statement & {\it var-id} {\tt =}  expr {\tt ;} \\
            & {\it var-id} {\tt +=} expr {\tt ;} \\
            & {\tt function} {\it id} {\tt (} formals {\tt ) = } expr {\tt ;} \\
            & iospec decls {\tt ;}
\vspace{5mm} \\
  iospec    & {\tt input} \gOR\ {\tt output} \gOR\ {\tt inout} \gOR\
              {\tt complex input} \gOR\ {\tt complex inout} \\
  decls     & decl initializer {\tt ,} decl initializer {\tt ,} $\ldots$ \\
  decl      & {\it var-id} \gOR\ 
              {\it var-id} {\tt (} {\it m} {\tt )} \\
            & {\it var-id} {\tt (} {\it m} {\tt ,} {\it n} {\tt )} \gOR\
              {\it var-id} {\tt symmetric} {\tt (} {\it m} {\tt )} \gOR\
              {\it var-id} {\tt [} {\it lda} {\tt ]} 
                           {\tt (} {\it m} {\tt ,} {\it n} {\tt )} \\
  initializer & {\tt =} expr \gOR\ $\epsilon$ \\
  formals & {\it id} {\tt ,} {\it id} {\tt ,} $\ldots$ 
\vspace{5mm} \\
  expr & expr {\tt :} expr \\
       & expr {\tt +} expr \\
       & expr {\tt -} expr \\
       & expr {\tt *} expr \\
       & expr {\tt /} expr \\
       & {\tt -} expr \\
       & expr {\tt '} \\
       & {\tt (} expr {\tt )} \\
       & {\it var-id} \\
       & {\it number} \\
       & matrix \\
       & {\it func-id} {\tt (} expr {\tt ,} expr {\tt ,} $\ldots$ {\tt )} \\
       & {\it var-id} {\tt (} expr {\tt )} \gOR\ 
         {\it var-id} {\tt (} expr {\tt ,} expr {\tt )} \\
  matrix  & {\tt [} rows {\tt ]} \\
  rows    & row {\tt ;} row {\tt ;} $\ldots$ \\
  row     & expr {\tt ,} expr {\tt ,} $\ldots$
\end{tabular}
\caption{\matexpr\ call syntax}
\label{matexpr-syntax-fig}
\end{center}
\end{figure}


\section{Array handling}

Matrices are represented as C arrays, but with Fortran-style
column-major storage.  Input arrays can be declared symmetric, in
which case only the upper triangle is accessed; a matrix declared as
complex and symmetric is {\em not} Hermitian.  An array used for input
or output can be specified with a leading dimension given in brackets;
this is used, for example, to pass submatrices into \matexpr-generated
expressions.  The array dimensions and the leading dimension must all
be integer constants.

Expressions of the form $A(i)$ or $A(i,j)$ where $A$ is an array are
interpreted as subscript operations.  At present, the subscripts
\emph{must} be compile-time integer constants.  If only one index is given
for a two-dimensional array, it is interpreted as the index when the
entries are listed in column-major order.  Indexing is one-based.


\section{Functions}

If \matexpr\ sees an expression of the form $f(...)$, where $f$ is not
known to be a variable, it interprets the expression as a function
call.  If $f$ corresponds to a declared function name, the function is
called inline; if it is a special function, it is handled
appropriately; and otherwise, it is interpreted as a C function call.
If $f$ is known to be a variable, the expression is interpreted as a
subscript operation.

\matexpr\ recognizes two special functions:
\begin{itemize}
\item 
  {\tt deriv(f, x)} -- differentiate the function $f$ with respect
  to the input variable(s) $x$.  The second argument can be a
  matrix; for example, {\tt deriv(f, [x, y])} is equivalent to
  {\tt [ deriv(f, x), deriv(f, y) ]}.  Similarly,
  {\tt deriv(f, [x; y])} is equivalent to {\tt [deriv(f,x); deriv(f,y)]}.
  \matexpr\ only does forward-mode differentiation, and only handles
  basic arithmetic operations and a few elementary transcendental functions.
\item
  {\tt eye(n)} -- produce an $n$-by-$n$ identity matrix.  $n$ must be a
  compile time constant.
\end{itemize}

For C functions, \matexpr\ currently only allows functions of one
argument.  If the argument specified is a matrix, \matexpr\ evaluates
the function elementwise.


\end{document}
