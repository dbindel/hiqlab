\section{Introduction}
In this manual several examples are presented to 
introduce and explain the functions and features of
\ttt{HiQLab}. The examples are complimentary to the 
User's manual and each coexist to clarify software.
Each example is aimed at introducing a feature or
function of \ttt{HiQLab} in a tutorial like fashion.
By working through each example,  the User will be
fully accustomed to the full capabilities of the 
software.

Simulation in \ttt{HiQLab} can be conducted in 
either the \ttt{Lua} environment which is standalone
but lacks visualization capabilities,
or the \ttt{MATLAB} environment which allows one to
use the various powerful functions of the software.

The solution process for the \ttt{MATLAB} environment
can be divided into the following two steps.
\begin{enumerate}
\item Construct input files in \ttt{Lua} which describe
      the geometry of the mesh and define functions for 
      analysis.
\item Load \ttt{Lua} input file into the \ttt{MATLAB}
      environment, solve the problem, and visualize the 
      results..
\end{enumerate}
In the \ttt{Lua} standalone version, the first step is
identical. The second step, instead is conducted again
in \ttt{Lua}. The visualization is still under construction
for this version.
