\newpage
\section{Basic analysis (Lua)}
\subsection{Functions to obtain basic information about mesh}
\begin{codelist}

  \item[ndm = Mesh:get\_ndm()]
  Return the dimension of the ambient space for the mesh.

  \item[numnp = Mesh:numnp()]
  Return the number of nodal points in the mesh.

  \item[ndf = Mesh:get\_ndf()]
  Return the number of degrees of freedom per node in the mesh.

  \item[numid = Mesh:get\_numid()]
  Return the total number of degrees of freedom in the mesh.

  \item[numelt = Mesh:numelt()]
  Return the number of elements in the mesh.

  \item[nen = Mesh:get\_nen()]
  Return the maximum number of nodes per element.

  \item[numglobals = Mesh:numglobals()]
  Return the number of global degrees of freedom in the mesh

  \item[nbranch\_id = Mesh:nbranch\_id()]
  Return the number of branch variables in the mesh

\end{codelist}

\subsection{Obtain particular information about ids}
\begin{codelist}

  \item[id = Mesh:id(mesh,i,j)]
  Return the ith variable of node j

  \item[id = Mesh:branchid(mesh,i,j)]
  Return the ith variable of branch j

  \item[id = Mesh:globalid(mesh,j)]
  Return the id for jth global variable

  \item[nbranch\_id = Mesh:nbranch\_id(mesh,j)]
  Return the number of variables of branch j

\end{codelist}





%\begin{verbatim}
%- plotmesh(mesh) to plot the mesh what boundary codes mean
%-- F =-Mesh_assemble_R(mesh);
%-- K = Mesh_assemble_k(mesh);
%%-- U = K\F;
%-- Mesh_set_u(mesh, U);
%-- disp = Mesh_get_disp(mesh);
%-- plotfield2d(mesh);
%-- Mesh_delete(mesh);
%-- Mesh_get_id(mesh);
%-- How do you compute stresses???
%\end{verbatim}
%\subsection{Static analysis}
%\subsection{Modal analysis}
%\subsection{Time-harmonic analysis}
%\subsection{Model reduction}
