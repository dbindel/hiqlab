\newpage
\section{Basic features from Lua}
\emph{Write about and or features!!}

\subsection{Variable types}
Since Lua is a dynamically typed language, there
are no type definitions in the language. In other
words, the type is defined by the variable assigned. 
There are eight basic types in Lua. Of these the
{\tt HiQLab} user should be aware of the following 6.
\begin{enumerate}
\item {\tt nil}: This is the type that is assigned to all variables
by default. {\tt nil} can be assigned to a variable as,
\begin{verbatim}
   a = nil
\end{verbatim}
\item {\tt boolean}: This has two types, {\tt true}and {\tt false}. 
In Lua, any value may represent a condition. In conditionals,
\emph{ONLY} {\tt false} and {\tt nil} are considered false, and everything
else is considered true. Beware that the value zero and empty string
both represent \emph{TRUE}. 
\item {\tt number}: Lua has only one number type, real double-precision
floating point numbers. There are \emph{NO} integer types. Valid types
of number representations are,
\begin{verbatim}
            4     0.4    4.57e-3   0.3e12   5e+20
\end{verbatim}
\item {\tt string}: Strings have the usual meaning, a sequence of 
characters. The string can be assigned by putting them between 
single quotes or double quotes. They can be assigned to a variable 
by the following statement.
\begin{verbatim}
   a = "a line"
   b = 'another line'
\end{verbatim}
Strings in Lua can contain the following C-like escape sequences:
\begin{table}[htbp]
\caption{C-like escape sequences}
\centering
\begin{tabular}{c|c}
{\tt \textbackslash b} & : back space \\
{\tt \textbackslash n} & : newline \\
{\tt \textbackslash t} & : horizontal tab \\
{\tt \textbackslash v} & : vertical tab \\
{\tt \textbackslash\textbackslash} & : back slash \\
{\tt \textbackslash '} & : single quote \\
{\tt \textbackslash "} & : double quote 
\end{tabular}
\end{table}

Strings can be concatenated by the operator {\tt ..}.
\begin{verbatim}
   a = "Hello"
   b = "World"
   c = a..b
   d = a.." "..b
   print(c)         --> HelloWorld
   print(d)         --> Hello World
\end{verbatim}

\item {\tt table}: This type is used to implement associative arrays.
An associative array is an array that can be indexed not only with
numbers, but also with strings or any other value of the language. 
Moreover, tables have no fixed size, and the size is adjusted dynamically.
Thus a single Lua table can contain different types of data.
\begin{verbatim}
   a    = {}               -- create a table
   a[1] = 4                -- store double
   a[21]= 'Hello world'    -- store string
   print(a[1])             --> 4

   a['A']    = a           -- Index with character
   a['John'] = Doe         -- Index with string
   print(a['John'])        --> Doe
   print(a[John])          --> nil
\end{verbatim}
Tables in Lua are treated as objects similar to Java. Thus a
program that manipulates tables, only manipulates references
or pointer to them. 
\begin{verbatim}
   b =  a                  -- the reference to the table that 'a'
                          -- points to is passed
\end{verbatim}

Additionally, since tables are like objects,
{\tt a.name} cna be used as syntactic sugar for {\tt a["name"]}.
\begin{verbatim}
   a      = {}
   a["x"] = 4 
   print(a["x"])           --> 4
   print(a.x)              --> 4
\end{verbatim}
Tables can be initialied with values by the following argument,
in which case the keys to the corresponding values start from
one (and not with zero, as in C).
\begin{verbatim}
   a      = { 10, 11, 12}
   print(a[1])             --> 10
\end{verbatim}

\item {\tt function} This aspect will further be explained in
section ????.

\end{enumerate}

WHAT YOU SHOULD KNOW(Zentei Chishiki).

\subsection{Ending a line}
A statement in Lua is called a {\tt chunk},
and is simply a sequence of statements. 
This {\tt chunk} can take a single line,
multiple lines, or can even span multiple files.
A semicolon may optionally follow any statement,
but this is just a convention. Thus, the following
four chunks are equivalent.
\begin{verbatim}
   -- Chunk 1
   a = 1
   b = a*2

   -- Chunk 2
   a = 1;
   b = a*2

   -- Chunk 3
   a = 1; b = a*2

   -- Chunk 4
   a = 1  b = a*2
\end{verbatim}


\subsection{Print statement}
All variable types can be printed to the screen through
the command {\tt print}
\begin{verbatim}
   a = 4
   print(a)     --> 4
   b = "Hello World"
   print(b)     --> Hello World
   c = false
   print(c)     --> false
   d = {4, 5, 6}
   print(d)     --> table: 0x8067d90
                 (The reference is printed in this case)
   print(d[2])  --> 5
\end{verbatim}


\subsection{Comments}
A comment starts anywhere with a double hyphen (--)
and runs until the end of the line. Lua also offers
block comments, starting with {\tt --[[} and run till
{\tt --]]}.
\begin{verbatim}
   -- This is a commented line

   --[[
      print(10)             --This is a commented block
   --]]
\end{verbatim}

\subsection{For loops}
\ttt{for} loops in Lua are stated by the following 
structure.
\begin{verbatim}
   for var=start_value, end_value, increment do
       do something
   end
\end{verbatim}
The loop will execute \ttt{something} for each value
of \ttt{var} from \ttt{start\_value} to \ttt{end\_value}
with and increment of \ttt{increment}. If \ttt{increment}
is absent, the increment will be assumed one. 

An example for printing the numbers 1 through 10 is 
presented below.  
\begin{verbatim}
for i=1,10 do
  print(i)  
end
\end{verbatim}
One remark that should be made is that in the example
above the index $i$ is declared as a local variable.
This implies that once the for loop is terminated the
value of $i$ will not be retained and cannnot be referenced
to.

\subsection{If statements}
\ttt{if} statements in Lua are defined by the following 
structures.
\begin{verbatim}
   if condition_expression then
      do something
   end
\end{verbatim}
or,
\begin{verbatim}
   if condition_expression then
      do something
   else
      do something else
   end
\end{verbatim}
All other values other than \ttt{false} or \ttt{nil} that
are returned by the \ttt{condition\_expression} will be 
treated as \ttt{true}.

An example code to print the larger of the value \ttt{a,b}
is:
\begin{verbatim}
   if a > b then
      print(a)
   else
      print(b)
   end
\end{verbatim}

To avoid writing nested \ttt{if}s, one can use
\ttt{elseif} in the following structure.
\begin{verbatim}
   if condition_expression then
      do something
   elseif condition_expression
      do something else
   else
      do yet something else
   end
\end{verbatim}

\subsection{Functions}
A function in Lua is defined by the following 
format.
\begin{verbatim}
   function function_name(input_arguments)
      function_body
   end
\end{verbatim}

The input arguments can any type of Lua variable,
and when multiple values are passed, they must be 
seperated by a comma. It is possible to assign
no input arguments.

\ttt{function\_body} will consist of the standard
Lua chunks. For the function to return an output 
value the \ttt{return} command must be placed.
\begin{verbatim}
   function function_name(input_arguments)
      function_body
      return output_arguments
   end
\end{verbatim}
\ttt{output\_arguments} may return any type of Lua
variable, and when multiple values are returned, they
must be seperated by a comma.

An example of a function which takes two numbers \ttt{a,b}
as the input and returns their sum is 
shown below.
\begin{verbatim}
   function add(a,b)

      sum_ab = a + b

      return sum_ab
   end
\end{verbatim} 
The method of calling this function is the following.
\begin{verbatim}
   a = 4
   b =-2.11
   sum_ab = add(a,b)
   print(sum_ab)      -> 1.89
\end{verbatim}
The function above can be modified to also return the difference
by the following code.
\begin{verbatim}
   function add_diff(a,b)

      sum_ab = a + b
      diff_ab= a - b

      return sum_ab, diff_ab
   end
\end{verbatim} 
The method of calling this function is the following.
\begin{verbatim}
   a = 4
   b =-2.11
   sum_ab, diff_ab = add_diff(a,b)
   print(sum_ab)      -> 1.89
   print(diff_ab)     -> 6.11
\end{verbatim}

