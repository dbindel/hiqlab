\newpage
\section{Basic functions (Matlab)}

\subsection{Loading and deleting Lua mesh input files}
The Lua object interface is used in the MATLAB \ttt{Mesh\_load}
function:
\begin{codelist}

  \item[Mesh\_load(filename,p)]
    Creates a Lua interpreter and executes the named file in order to
    generate a mesh object (which is returned).  The mesh should be
    named ``mesh''; if such an object is undefined on output,
    \ttt{Mesh\_load} returns an error message.  Before executing
    the named file, \ttt{Mesh\_load} copies the entries of the
    structure \ttt{p}, which may only be strings or doubles, into
    the Lua global state; in this way, it is possible to vary mesh
    parameters from MATLAB.

  \item[Mesh\_delete(mesh)]
    Deletes the mesh object.

\end{codelist}

\subsection{Getting Scaling parameters}
\begin{codelist}

  \item[scale\_param = Mesh\_get\_scale(scale\_name)]
  Get a characteristic scale described by the given name.

  \item[{[cu,cf]} = Mesh\_get\_scales(mesh)]
  Return an array of dimensional scales for the problem.
\begin{verbatim}
  Outputs:
   cu - scales for the primary variables
   cf - scales for the secondary (flux) variables
\end{verbatim}

\end{codelist}

\subsection{Obtain basic information about mesh}
\begin{codelist}

  \item[ndm = Mesh\_get\_ndm(mesh)]
  Return the dimension of the ambient space for the mesh.

  \item[numnp = Mesh\_numnp(mesh)]
  Return the number of nodal points in the mesh.

  \item[ndf = Mesh\_get\_ndf(mesh)]
  Return the number of degrees of freedom per node in the mesh.

  \item[numid = Mesh\_get\_numid(mesh)]
  Return the total number of degrees of freedom in the mesh.

  \item[numelt = Mesh\_numelt(mesh)]
  Return the number of elements in the mesh.

  \item[nen = Mesh\_get\_nen(mesh)]
  Return the maximum number of nodes per element.

  \item[numglobals = Mesh\_numglobals(mesh)]
  Return the number of global degrees of freedom in the mesh

  \item[nbranch\_id = Mesh\_nbranch\_id(mesh)]
  Return the number of branch variables in the mesh

  \item[x = Mesh\_get\_x(mesh, (cL))]
  Return an ndm-by-numnp array of node positions.
\begin{verbatim}
  Inputs:
  cL - characteristic length used for redimensionalization
        (default: mesh 'L' scale)
\end{verbatim}

  \item[e = Mesh\_get\_e(mesh)]
  Return the element connectivity array (a maxnen-by-numelt array).

  \item[id = Mesh\_get\_id(mesh)]
  Return the variable-to-identifier mapping (a maxndf-by-numnp array).
  Nodal variables subject to displacement BCs are represented by a 0.

  \item[bc = Mesh\_get\_bc(mesh)]
  Return an ndf-by-numnp array of boundary codes.  The codes are
\begin{verbatim}
  0 - No boundary condition for this dof
  1 - Displacement (essential) boundary conditions
  2 - Flux (natural) boundary conditions
\end{verbatim}

  \item[{[p,e,id,bc,numnp]} = Mesh\_get\_parameters(mesh,(cL))]
\begin{verbatim}
  Input:
  cL    - characteristic length to redim node positions
          (default: mesh 'L' parameter)
  Return mesh parameters:
  p     - Node position array
  e     - Element connectivity array (nen-by-numelt)
  id    - Variable identifier array (ndf-by-numnp)
  bc    - Boundary condition codes (see Mesh\_get\_bc)
  numnp - Number of nodal points
\end{verbatim}

\end{codelist}

\subsection{Obtain particular information about ids}
\begin{codelist}

  \item[ix = Mesh\_ix(mesh,i,j)]
  Return the ith node of element j

  \item[id = Mesh\_id(mesh,i,j)]
  Return the ith variable of node j

  \item[id = Mesh\_branchid(mesh,i,j)]
  Return the ith variable of branch j

  \item[id = Mesh\_globalid(mesh,j)]
  Return the id for jth global variable

  \item[nbranch\_id = Mesh\_nbranch\_id(mesh,j)]
  Return the number of variables of branch j

\end{codelist}

\subsection{Obtain particular information about id, nodes, or elements}
\begin{codelist}

  \item[x = Mesh\_x(mesh,i,j,(cL))]
  Return the ith coordinate of node j.
\begin{verbatim}
  Inputs:
   i  - coordinate index
   j  - node number
   cL - characteristic length scale for redimensionalization
        (default: mesh 'L' scale)
\end{verbatim}

  \item[nen = Mesh\_get\_nen\_elt(mesh,j)]
  Return the number of nodes for element j

\end{codelist}

\subsection{Getting displacements and force}
\begin{codelist}

  \item[{[u]} = Mesh\_get\_disp(mesh,is\_dim)]
  Get the node displacement array (dimensionless)
\begin{verbatim}
  Inputs:
   - is\_dim - should the vector be redimensionalized? (default: 1)
\end{verbatim}

  \item[{[f]} = Mesh\_get\_force(mesh,is\_dim)]
  Get the node force array (dimensionless)
\begin{verbatim}
  Inputs:
   - is\_dim - should the vector be redimensionalized? (default: 1)
\end{verbatim}

  \item[{[u]} = Mesh\_get\_u(mesh, cx, cv, ca, reduced, is\_dim)]
  Get the node displacement array (dimensionless)
\begin{verbatim}
  Inputs:
   - cx  For variables (default: 1)
   - cv  For 1st derivative variables
   - ca  For 2nd derivative variables
   - reduced or not?
\end{verbatim}

  \item[Mesh\_set\_u(mesh, u, v, a)]
  Set the displacement, velocity, and acceleration vectors.
  The vectors should be in dimensionless form.
\begin{verbatim}
  Inputs:
   u - displacement vector (or zero or no arg to clear the mesh u)
   v - velocity vector     (or zero or no arg to clear the mesh v)
   a - acceleration vector (or zero or no arg to clear the mesh a)
\end{verbatim}

\end{codelist}


\subsection{Functions to form global matrices}
\begin{codelist}

  \item[{[K]} = Mesh\_assemble\_k(mesh)]
  Assemble the stiffness matrix for the mesh.
  Matrix is in dimensionless form.

  \item[{[M,K]} = Mesh\_assemble\_mk(mesh)]
  Assemble the mass and stiffness matrices for the mesh.
  Matrices are in dimensionless form.

  \item[{[M,K,C]} = Mesh\_assemble\_mkc(mesh)]
  Assemble the mass, stiffness, and damping matrices.
  Matrices are in dimensionless form.

  \item[{[K]} = Mesh\_assemble\_dR(mesh, cx, cv, ca, reduced)]
  Assembles mass, damping, or stiffness matrix from mesh.
\begin{verbatim}
  If 'reduced' is given     -'reduced=1' Reduced form
                           -'reduced=0' Non-reduced form
\end{verbatim}
  Matrices are in dimensionless form.

  \item[[K] = Mesh\_element\_dR(mesh, eltid, cx, cv, ca)]
  Gets scattered mass, damping, or stiffness matrix from mesh element.
  Matrices are in dimensionless form.

  \item[{[R,Ri] = Mesh\_assemble\_R(mesh)}]
  Assemble the system residual vector and return it in R and Ri.
  The residual vector is in dimensionless form.
\begin{verbatim}
  Outputs:
    R  - real part of the residual vector
    Ri - imag part of the residual vector
\end{verbatim}

\end{codelist}

\subsection{Other useful functions}
\begin{codelist}
  \item[ E = Mesh\_mean\_power(mesh)]
  Compute the time-averaged energy flux at each node.
  TODO: The flux is currently in dimensionless form, but it probably
   shouldn't be.

  \item[f = Mesh\_get\_lua\_fields(mesh, name, nfields)]
  Return an array of field values.
\begin{verbatim}
  Inputs:
   name      - the name of the Lua function to define the fields
   nfields   - number of fields requested
\end{verbatim}

  \item[Mesh\_make\_harmonic(mesh,omega)]
  Set v = i*omega*u and a = -omega**2*u
\begin{verbatim}
  Input:
   omega - forcing frequency
   units - specify units of forcing frequency (default 'rs'):
     'hz': omega is in units of Hz
     'rs': omega is in units of rad/s
     'nd': omega is in dimensionless units
      cT : nondimensionalize omega using the given characteristic time
\end{verbatim}

\end{codelist}

\subsection{Assigning and reassigning ids}
\begin{codelist}
  \item[int = Mesh\_assign\_ids(mesh)]
  Number the degrees of freedom in the mesh.
  Returns the total number of dofs.

  \item[int = ted\_block\_mesh(mesh)]
  Relabel the nodal degrees of freedom so that all mechanical degrees of
  freedom come first, followed by all thermal degrees of freedom.  Return
  the total number of mechanical degrees of freedom.

  \item[int = pz\_block\_mesh(mesh)]
  Relabel the nodal degrees of freedom so that all mechanical degrees of
  freedom come first, followed by all potential degrees of freedom.  Return
  the total number of mechanical degrees of freedom.

\end{codelist}

\subsection{Producing forcing and sensing pattern vectors}
\begin{codelist}

  \item[u = Mesh\_get\_sense\_u(mesh, name, is\_reduced)]
  Return a vector for a displacement sense pattern.
  The vector is in dimensionless form.
\begin{verbatim}
  Inputs:
    name       - the name of the Lua function to define the pattern
   is\_reduced - do we want the reduced (vs full) vector?  Default: 1
\end{verbatim}

  \item[u = Mesh\_get\_sense\_disp(mesh, name, is\_reduced)]
  Return a vector for a displacement sense pattern.
  The vector is in dimension form.
\begin{verbatim}
  Inputs:
   name       - the name of the Lua function to define the pattern
   is\_reduced - do we want the reduced (vs full) vector?  Default: 1
\end{verbatim}

  \item[f = Mesh\_get\_sense\_f(mesh, name, is\_reduced)]
  Return a vector for a force sense pattern
  The vector is in dimensionless form.
\begin{verbatim}
  Inputs:
   name       - the name of the Lua function to define the pattern
   is\_reduced - do we want the reduced (vs full) vector?  Default: 1
\end{verbatim}

  \item[f = Mesh\_get\_drive\_f(mesh, name, is\_reduced)]
  Return a vector for a drive sense pattern
  The vector is in dimensionless form.
\begin{verbatim}
  Inputs:
   name       - the name of the Lua function to define the pattern
   is\_reduced - do we want the reduced (vs full) vector?  Default: 1
\end{verbatim}

  \item[f = Mesh\_get\_sense\_force(mesh, name, is\_reduced)]
  Return a vector for a force sense pattern
  The vector is in dimension form.
\begin{verbatim}
  Inputs:
   name       - the name of the Lua function to define the pattern
   is\_reduced - do we want the reduced (vs full) vector?  Default: 1
\end{verbatim}

  \item[u = Mesh\_get\_sense\_globals\_u(mesh, name)]
  Return a vector for a global variable displacement sense pattern.
  The vector is in dimensionless form.
\begin{verbatim}
  Inputs:
    name       - the name of the Lua function to define the pattern
\end{verbatim}

  \item[f = Mesh\_get\_sense\_globals\_f(mesh, name)]
  Return a vector for a global variable force sense pattern.
  The vector is in dimensionless form.
\begin{verbatim}
  Inputs:
    name       - the name of the Lua function to define the pattern
\end{verbatim}

  \item[f = Mesh\_get\_drive\_globals\_f(mesh, name)]
  Return a vector for a global variable force drive pattern.
  The vector is in dimensionless form.
\begin{verbatim}
  Inputs:
    name       - the name of the Lua function to define the pattern
\end{verbatim}

  \item[u = Mesh\_get\_sense\_elements\_u(mesh, name)]
  Return a vector for an element variable displacement sense pattern.
  The vector is in dimensionless form.
\begin{verbatim}
  Inputs:
    name       - the name of the Lua function to define the pattern
\end{verbatim}

  \item[f = Mesh\_get\_sense\_elements\_f(mesh, name)]
  Return a vector for an element variable force sense pattern.
  The vector is in dimensionless form.
\begin{verbatim}
  Inputs:
    name       - the name of the Lua function to define the pattern
\end{verbatim}

  \item[f = Mesh\_get\_drive\_elements\_f(mesh, name)]
  Return a vector for an element variable force drive pattern.
  The vector is in dimensionless form.
\begin{verbatim}
  Inputs:
    name       - the name of the Lua function to define the pattern
\end{verbatim}

\end{codelist}

\subsection{Getting and setting variables in the Lua environment}
\begin{codelist}

  \item[Lua\_set\_string(L, string\_name, string\_value)]
  Set a string variable in the Lua environment
\begin{verbatim}
  L            - the Lua interpreter object
  string\_name - the name of the Lua variable(must be string)
  s    - a string value(must be string)
\end{verbatim}

  \item[string = Lua\_get\_string(L, string\_name)]
  Get a string variable out of the Lua environment
\begin{verbatim}
  L    - the Lua interpreter object.
  name - the name of the Lua variable.
\end{verbatim}
  Returns an empty string if no such variable exists.

  \item[Lua\_set\_double(L, double\_name, double\_value)]
  Set a numeric variable in the Lua environment
\begin{verbatim}
  L    - the Lua interpreter object
  name - the name of the Lua variable
  x    - a number
\end{verbatim}

  \item[{[x]} = Lua\_get\_double(L, name)]
  Get a numeric variable out of the Lua environment
\begin{verbatim}
  x  - the value of the Lua variable   
\end{verbatim}

  \item[Lua\_set\_table\_double(L, table\_name, double\_name, double\_value x)]
  Set a numeric variable in a table in the Lua environment
\begin{verbatim}
  L     - the Lua interpreter object
  table\_name  - the name of the Lua table
  double\_name - the name of the Lua key
  double\_value- a number
\end{verbatim}
  key can be a number or a string.

  \item[{[x]} = Lua\_get\_table\_double(L, table\_name, double\_name)]
  Get a numeric variable out of a table in the Lua environment
\begin{verbatim}
  x  - the value of the Lua variable   
\end{verbatim}
  key can be a number or a string.

  \item[Lua\_set\_table\_string(L, table\_name, string\_name, string\_value)]
  Set a string variable in a table in the Lua environment
\begin{verbatim}
  L     - the Lua interpreter object
  table\_name  - the name of the Lua table
  string\_name - the name of the Lua key
  string\_value- a number
\end{verbatim}
  key can be a number or a string.

  \item[{[s]} = Lua\_get\_table\_string(L, table\_name, string\_name)]
  Get a string variable out of a table in the Lua environment
\begin{verbatim}
  s  - the string value of the Lua variable   
\end{verbatim}
  key can be a number or a string.

  \item[Lua\_set\_array(L, aname, array, a\_type)]
  Set a matrix variable in the Lua environment
\begin{verbatim}
  Input:
   L        - the Lua interpreter object
   aname    - the name of the matrix variable
   array    - a numeric array
   a\_type(0)- the type of array to construct
             0: Real array
             1: Real array of twice the size
                 (Not supported yet)
             2: Complex array
\end{verbatim}

  \item[{[m\_Object]} = Lua\_get\_array(L, name)]
  Get a matrix variable from the Lua environment
\begin{verbatim}
  Input:
  L     - the Lua interpreter object
  name  - the name of the Lua variable
\end{verbatim}

\end{codelist}

\subsection{Manipulating the Lua environment}
The same interfaces that are automatically bound to Lua are also
automatically bound to MATLAB.  In addition, several methods are
defined which allow MATLAB to manipulate a Lua interpreter:
\begin{codelist}

  \item[Lua\_open]      
    Return a pointer to a new Lua interpreter \ttt{L}

  \item[Lua\_close(L)]  
    Close the Lua interpreter

  \item[Lua\_dofile(L,filename)]
    Execute a Lua file

  \item[Lua\_set\_mesh(L,name,mesh)]
    Assign a mesh object to a Lua global

  \item[Lua\_get\_mesh(L,name)]
    Retrieve a mesh object from a Lua global

\end{codelist}

\clearpage
\section{Functions for analysis(MATLAB)}
\subsection{Static analysis}
  Solves for the static state of a device. The equation,
\begin{eqnarray}
  \bfR(\bfU) = 0
\end{eqnarray}
  for the general case, and 
\begin{eqnarray}
\bfK\bfU = \bfF
\end{eqnarray}
  for the linear case is solved. 

\begin{codelist}

  \item[static\_state(mesh,opt)]
  Solves for the static state. If the field nonlinear is 
  not specified only 1 iteration will be conducted.
\begin{verbatim}
 Input: mesh  - Mesh object
       *opt
         nonlinear('NR')- Conduct non-linear solve
                       'NR' :Newton-Raphson
                       'MNR':Modified-Newton-Raphson
         kmax(20)      - Max number of iterations
         du_tol(1e-15) - Tolerance for convergence
                         for increment
         R_tol (1e-15) - Tolerance for convergence
                         for residual
         U             - Initial starting vector for solve
\end{verbatim}

\end{codelist}

\subsection{Time-harmonic analysis}
Solves for the time harmonic state.
\begin{eqnarray}
\left(\bfK + i\omega \bfC - \omega^2\bfM\right)\bfu = \bfF
\end{eqnarray}

\begin{codelist}

 \item[harmonic\_state(mesh,F,w,opt)]
 Solves for the time-harmonic state.
\begin{verbatim}
 Input: mesh     - Mesh object
        F        - Forcing vector(in nondimensional form)
        w        - Forcing frequency
       *opt
         mkc(0)        - Include damping matrix
         kmax(1)       - Max number of iterations
         du_tol(1e-15) - Tolerance for convergence
                        for increment
         R_tol(1e-15)  - Tolerance for convergence
                        for residual
\end{verbatim}

\end{codelist}

\clearpage
\subsection{Modal analysis}
The MATLAB sparse eigensolver routine \ttt{eigs} is actually an
interface to ARPACK (see Section~\ref{section-eigs}).  We express all
frequencies in radians/s rather than Hertz.  We provide one function
to compute complex frequencies and associated mode shapes for PML
eigenvalue problems:
\begin{codelist}

  \item[{[V,w,Q]} = pml\_mode(M,K,w0,nmodes,opt)]
    Find the requested number of modes closest in frequency to
    \ttt{w0}.  Return an array of mode shapes, a vector of complex
    frequencies, and a vector of $Q$ values.  Options are 
    \begin{codelist}[use\_umfpack]
      \item[use\_matlab]  Use MATLAB's eigs rather than ARPACK?
      (default: false)
      \item[use\_umfpack]  Use UMFPACK with MATLAB eigs, if present?
      (default: true)
      \item[disp]  Verbosity level? (default: 0)
    \end{codelist}

   \item[{[V,w,Q]} = tedmode(mesh, w0, nev, opt)]
   Computes the eigenfrequencies and modes of a thermoelastic problem.
\begin{verbatim}

 Compute nev complex frequencies w (in rad/s) near target frequency
 w0, and also associated Q values.  V contains the eigenvectors
 (mechanical and thermal parts).

 *Note*: tedmode will reorder the degrees of freedom in the mesh.

 opt   contains optional parameters:
   mech - Conduct purely mechanical analysis
   type - Use a perturbation method ('pert') or linearization
          ('full').  Default is 'full'.
   T0   - Baseline temperature for use in symmetric linearization
   cT   - Characteristic time scale for redimensionalization
          (Default: Mesh_get_scale('T')
   use_matlab - use matlab eigs or not (Default:0)
\end{verbatim}

   \item[{[V,w,Q]} = emmode(mesh, w0, nev, opt)]
   Computes the eigenfrequencies and modes of a electromechanical problem.
\begin{verbatim}
 Compute nev complex frequencies w (in rad/s) near target frequency
 w0, and also associated Q values.  V contains the eigenvectors


 opt   contains optional parameters:
   mech - Conduct purely mechanical analysis
   type - Use a perturbation method ('pert') or linearization
          ('full').  Default is 'full'.
   cT   - Characteristic time scale for redimensionalization
          (Default: Mesh_get_scale('T')
   use_matlab - use matlab eigs or not (Default:1)
   idg_m - array of global numbers for mechanical variables
   idg_p - array of global numbers for potential variables

\end{verbatim}

   \item[{[V,w,Q]} = emcmode(mesh, w0, nev, opt)]
   Computes the eigenfrequencies and modes of a electromechanical problem with 
   surrounding circuitry.
\begin{verbatim}
 Compute nev complex frequencies w (in rad/s) near target frequency
 w0, and also associated Q values.  V contains the eigenvectors


 opt must contain following parameters
   eno - array of element numbers for electrodes
   idg - array of global number for driving electrode

 opt   contains optional parameters:
   mech - Conduct purely mechanical analysis
   type - Use a perturbation method ('pert') or linearization
          ('full').  Default is 'full'.
   cT   - Characteristic time scale for redimensionalization
          (Default: Mesh_get_scale('T')
   use_matlab - use matlab eigs or not (Default:1)

\end{verbatim}

\end{codelist}

\clearpage
\subsection{Transfer function evaluation}
\begin{codelist}

  \item[{[H,freq]} = second\_order\_bode(mesh,wc,drive\_pat,sense\_pat,opt)]
  Computes the transfer function given the driving pattern and sensing pattern.
  The frequency range is set by specifying the center frequency \ttt{wc} and
  the range. Model reduction can be conducted by specifying the number of iterations
  \ttt{kmax} conducted to produce the reduced order model. Different projection 
  bases can also be selected.
\begin{verbatim}

   Output: freq          - Frequency array [rad/s]
           H             - Output
   Input:  mesh          - the mesh input
           wc            - center frequency[rad/s]
           drive_pat     - Driving pattern vector
           sense_pat     - Sensing pattern vector
          *opt
          -freq          - Predefined array of freq
          -mkc(0)        - Include damping or not
          -cT            - Characteristic time scale for
                           redimensionalization
                           (Default: Mesh_get_scale('T')
          -wr_min(0.90)  - left  value mag of bode plot
          -wr_max(1.10)  - right value mag of bode plot
          -w_ndiv(50)    - number divisions in bode plot
          -w_type('lin') - division type (linspae or logspace)
          -kmax(0)       - number of arnoldi iterations
          -realbasis(0)  - use real basis??
          -structurep(0) - use structure preserving basis??
          -use_umfpack   - use UMFPACK?? (Default: use if exist)
 If kmax and wc are given, use model reduction via an Arnoldi
 expansion about the shift w0.  
\end{verbatim}

\end{codelist}

\subsection{Model reduction}
There is currently one model reduction routine in the MATLAB support
files for \hiq.  As before, all frequencies are expressed in radians/s
rather than Hz.
\begin{codelist}

  \item[{[Mk,Dk,Kk,Lk,Bk,Vk]} = rom\_arnoldi(M,K,l,b,kk,w0,opt)] 
    Takes \ttt{kk} steps of
    shift-and-invert Arnoldi to form a basis for the Krylov subspace
    $\mathcal{K}_k\left( (K-(2 \pi \omega_0)^2 M)^{-1} M, b \right)$
    in order to form a reduced system to estimate the system transfer
    function.  Returns reduced matrices $M_k$, $K_k$, $l_k$, and
    $b_k$, along with the projection basis $V_k$.  If
    \ttt{opt.realbasis} is set to true (default is false), then the
    projection will use a real basis for the span of $[\Re(V_k),
    \Im(V_k)]$.  To do this, the matrix $[\Re(V_k), \Im(V_k)]$ will be
    orthonormalized using an SVD, and vectors corresponding to values
    less than \ttt{opt.realtol} (default $10^{-8}$) will be dropped.

   \item[{[Mk,Dk,Kk,Lk,Bk,Vk]} = rom\_soar(M,D,K,L,B,kk,w0,opt)]
    Takes \ttt{kk} steps of 
    shift-and-invert SOAR to form a basis for the second order 
    Krylov subspace,
    in order to form a reduced system to estimate the system transfer
    function.  Returns reduced matrices $M_k$, $K_k$, $l_k$, and
    $b_k$, along with the projection basis $V_k$.  If
    \ttt{opt.realbasis} is set to true (default is false), then the
    projection will use a real basis for the span of $[\Re(V_k),
    \Im(V_k)]$.  To do this, the matrix $[\Re(V_k), \Im(V_k)]$ will be
    orthonormalized using an SVD, and vectors corresponding to values
    less than \ttt{opt.realtol} (default $10^{-8}$) will be dropped.
    To form the structure preserving base for the thermoelastic problem,
    the id array must be reordered so that the mechanical dofs come first
    before the thermal dofs. \ttt{opt.structurep} must be set to 1, and 
    the number of mechanical degrees of freedom must also be specified.

\end{codelist}
