\newpage
\section{Global variables}
\label{section:GlobalVariables}
\subsection{Concept of a global variable}
Introducing a global variable can be understood as 
a means to specify constraints among individual 
nodal degrees of freedom much like a lagrange
multiplier. It allows one to specify a mode shape
for selected nodal degrees of freedom and assigns
a general degree of freedom for that mode.

The following simple example of a three degree of
freedom spring mass mechanical system may clarify
this concept. The equation of motion for the system
can be written as,
\begin{eqnarray}
\left[
\begin{array}{ccc}
m_1 & 0   & 0   \\
0   & m_2 & 0   \\
0   & 0   & m_3 
\end{array}
\right]
\left(
\begin{array}{c}
\ddot{u}_1(t) \\
\ddot{u}_2(t) \\
\ddot{u}_3(t)
\end{array}
\right)
+
\left[
\begin{array}{ccc}
k_1+k_2 & -k_2       &    0 \\
-k_2    &  k_2 + k_3 & -k_3 \\
-k_2    &  k_2 - k_3 &  k_3 
\end{array}
\right]
\left(
\begin{array}{c}
{u}_1(t) \\
{u}_2(t) \\
{u}_3(t)
\end{array}
\right)
=
\left(
\begin{array}{c}
{F}_1(t) \\
{F}_2(t) \\
{F}_3(t)
\end{array}
\right)
\end{eqnarray}
or in more compact form,
\begin{eqnarray}
\bfM\ddot{\bfu}(t) + \bfK\bfu(t) &=& \bfF(t)
\end{eqnarray}
If we assume that the 2 and 3 degrees of freedom
are connected by a rigid element, then the
degrees of freedom can be expressed by one 
general degree of freedom as follows,
\begin{eqnarray}
\left(
\begin{array}{c}
{u}_1(t) \\
{u}_2(t) \\
{u}_3(t)
\end{array}
\right)
&=& 
\left(
\begin{array}{c}
1  \\
0  \\
0 
\end{array}
\right)
u_1(t)
+
\left(
\begin{array}{c}
0 \\
1 \\
1
\end{array}
\right)
z(t) \nonumber\\
&=& 
\bfe_1 u_1(t) + \bfV_z z(t) \nonumber\\
&=&
\left[
\begin{array}{cc}
\bfe_1 & \bfV_z
\end{array}
\right]
\left(
\begin{array}{c}
u_1(t) \\
z(t) 
\end{array}
\right) \nonumber\\
&=& 
\bfW 
\left(
\begin{array}{c}
u_1(t) \\
z(t) 
\end{array}
\right) 
\end{eqnarray}
Inserting this expression and restricting the error to be 
orthogonal to this subspace spanned by the column vectors of 
$\bfW$ results in the two degree of freedom system,
\begin{eqnarray}
\bfW^*\bfM\bfW
\left(
\begin{array}{c}
\ddot{u}_1(t) \\
\ddot{z}(t) 
\end{array}
\right) 
+ \bfW^*\bfK\bfW 
\left(
\begin{array}{c}
u_1(t) \\
z(t) 
\end{array}
\right) 
&=& \bfW^*\bfF(t)
\end{eqnarray}
that must be solved. 

\subsection{Creating and defining global variables}
The key to creating a global variable is to construct the
shape vector $\bfV_z$ pertaining to the global variable.
This is conducted by specifying a function which is evaluated at
each nodal degree of freedom, similar to the form of the
nodal boundary conditions without the initial string. There
is also an additional restriction that a number must be assigned
for each nodal degree of freedom. As an example the shape
function for the first example would have the form,
\begin{verbatim}
      function shape_func_global_variable(x)
          if x==2 then return 1; end
          if x==3 then return 1; end
      end
\end{verbatim}

The general constructor for adding global variables to the
mesh is:
\begin{codelist}

  \item[index = Mesh:add\_global(shapeg,s\_1,s\_2)]
  \ttt{shapeg} must be a function of the form presented above.
  The variables \ttt{s\_1,s\_2} are optional and are the nondimensionalization
  parameters for the global variable. They can be numbers or strings
  that are predefined in the table \ttt{dim\_scales}. (See section on
  non-dimensionalization for details). The function returns the \ttt{index}
  of the global variable.

  \item[{Mesh:set\_globals()}]
  Once all the global variables required are added they must be 
  set to the mesh. Unless this function is called, the process will
  not be complete.

\end{codelist}
For the case presented in the previous section adding the global 
variable would take the process,
\begin{verbatim}
      index = mesh:add_global(shape_func_global_variable,'L','F')
      mesh:set_globals()
\end{verbatim}
The value for the strings \ttt{'L'} and \ttt{'F'} must be set
in \ttt{dim\_scales}.


