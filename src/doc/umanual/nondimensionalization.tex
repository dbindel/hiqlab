\clearpage
\section{Non-dimensionalization}
The solution to involving coupled fields can
cause computational difficulties to the various
physical time scales that existent. As a result
the magnitude of the entries in the matrices can
vary tremendously, leading to very ill-conditioned
matrices. To circumvent this problem, non-dimensionalization
of dimensional parameters is conducted. 

\subsection{Incorporating non-dimensionalization}
Non-dimensionalization can easily be incorporated in the
analysis by specifying the functions summarized in 
Table~\ref{table:FunctionsNonDimensionalConstants}
at the beginning of the input file immediately
after the mesh construction. The only thing that the
user must take care is on which functions return 
dimensionalized values and which do not. Generally,
values which are intermediate results are returned in
non-dimensionlized form and those that are the final
results are dimensionalized. For instance, an array
returning just the free degrees of freedom would be 
non-dimensionalized as opposed to an array of nodal 
degrees of freedom at every node in the mesh would be 
dimensionalized.  
These are summarized in Tables 
\ref{table:DimensionalizingFunctions} and
\ref{table:NondimensionalizingFunctions}.
Thus if values are extracted from the arrays \ttt{disp}
or \ttt{force} obtained from,
\begin{verbatim}
      disp = Mesh_get_disp(mesh);
      force= Mesh_get_force(mesh);
\end{verbatim}
they will be in their dimensional form. Also extraction
of desired quantaties by multiplying \ttt{disp} or \ttt{force}
with the sense vectors obtained from,
\begin{verbatim}
      sense_disp_vec = Mesh_get_sense_disp(mesh,'disp_func');
      sense_force_vec= Mesh_get_sense_force(mesh,'force_func');
\end{verbatim}
will also result in dimensional results.

Vectors \ttt{U} or \ttt{F} obtained through,
\begin{verbatim}
      U = Mesh_get_u(mesh);
      F = Mesh_assemble_k(mesh) * U;
\end{verbatim}
will contain only the free dofs in their nondimensional form.
But multiplication of these vectors with the sense vectors
obtained through,
\begin{verbatim}
      sense_u_vec = Mesh_get_sense_u(mesh,'u_func');
      sense_f_vec = Mesh_get_sense_f(mesh,'f_func');
\end{verbatim}
will give dimensional results. 

See example of a MEMS cantilever in the Examples manual for
details on how to use these functions. 

\begin{table}[htbp]
\caption{Functions to evaluate non-dimensional constants}
\label{table:FunctionsNonDimensionalConstants}
\vspace{0.1in}
\centering
\begin{tabular}{l|m{2in}|l|l}
\hline
\multicolumn{1}{c|}{\tbf{Function name}} & 
\multicolumn{1}{c|}{\tbf{Required fields}}&
\multicolumn{1}{c|}{\tbf{Key constants}} & 
\multicolumn{1}{c}{\tbf{Aux constants}} \\
\hline
\hline
\ttt{mech\_nondim(mtype,cL)} &
\ttt{mtype.E}, \ttt{mtype.nu}, \ttt{mtype.rho}, \ttt{cL} &
\ttt{M,L,T} &
\ttt{F} \\
\hline
\ttt{ted\_nondim(mtype,cL)} &
\ttt{mtype.E},  \ttt{mtype.nu}, \ttt{mtype.rho}, 
\ttt{mtype.at}, \ttt{mtype.cp}, \ttt{mtype.kt}, 
(\ttt{mtype.T0} or \ttt{dim\_scales.T0}), \ttt{cL} &
\ttt{M,L,T,Th} &
\ttt{F,Qt} \\
\hline
\ttt{pz\_nondim(mtype,cL)} & 
\ttt{mtype.E},  \ttt{mtype.nu}, \ttt{mtype.rho}, 
\ttt{mtype.kds}, \ttt{mtype.d}, \ttt{cL} &
\ttt{M,L,T,A} &
\ttt{F,V,Q,E,R} \\
\hline
\ttt{em\_nondim(mtype,cL,(eps))} &
\ttt{mtype.E},  \ttt{mtype.nu}, \ttt{mtype.rho}, 
\ttt{mtype.kds}, \ttt{mtype.d}, \ttt{cL}, 
(\ttt{eps} or \ttt{mtype.eps} or \ttt{dim\_scales.eps})&
\ttt{M,L,T,A} &
\ttt{F,V,Q,E,R} \\
\hline
\end{tabular}
\end{table}

\begin{table}[htbp]
\centering
\caption{Returns dimensional quantaties}
\label{table:DimensionalizingFunctions}
\begin{tabular}{|l|m{3in}|}
\hline
function name & value returned \\
\hhline{|=|=|}
\ttt{Mesh\_x.m}                &  A coordinate of a specific node \\
\ttt{Mesh\_get\_x.m}           &  Array of nodal coordinates      \\
\hline
\ttt{Mesh\_get\_disp.m}        &  Array of nodal displacements    \\
\ttt{Mesh\_get\_force.m}       &  Array of nodal forces           \\
\hline
\ttt{Mesh\_get\_sense\_disp.m} &  Sense vector for nodal displacements \\
\ttt{Mesh\_get\_sense\_force.m}&  Sense vector for nodal forces        \\
\hline
\end{tabular}
\end{table}
\begin{table}[htbp]
\centering
\caption{Returns non-dimensional quantaties}
\label{table:NondimensionalizingFunctions}
\begin{tabular}{|l|m{3in}|}
\hline
function name & value returned \\
\hhline{|=|=|}
\ttt{Mesh\_get\_u.m}           & Vector of displacement free dofs       \\
\ttt{Mesh\_set\_u.m}           & Set a vector of displacement free dofs \\
\hline
\ttt{Mesh\_assemble\_k.m}            &Assemble stiffness matrix \\
\ttt{Mesh\_assemble\_mk.m}           &Assemble mass and stiffnes  matrices \\
\ttt{Mesh\_assemble\_mkc.m}          &Assemble mass,stiffness, and damping matrices \\
\ttt{Mesh\_assemble\_dR.m}           &Assemble gradient of residual \\
\ttt{Mesh\_assemble\_R.m}            &Assemble residual\\
\hline
\ttt{Mesh\_get\_sense\_u.m}    & Sense vector for displacement free dofs\\
\ttt{Mesh\_get\_sense\_f.m}    & Sense vector for force free dofs       \\
\ttt{Mesh\_get\_drive\_f.m}    & Drive vector for force free dofs       \\
\ttt{Mesh\_get\_sense\_globals\_u.m} &Sense vector for displacement global variable free dofs\\
\ttt{Mesh\_get\_sense\_globals\_f.m} &Sense vector for force global variable free dofs \\
\ttt{Mesh\_get\_sense\_globals\_f.m} &Sense vector for force global variable free dofs \\
\ttt{Mesh\_get\_drive\_elements\_u.m}&Drive vector for displacement element variable free dofs\\
\ttt{Mesh\_get\_sense\_elements\_f.m}&Sense vector for force global variable free dofs \\
\ttt{Mesh\_get\_drive\_elements\_f.m}&Drive vector for force global variable free dofs \\
\hline
\end{tabular}
\end{table}


\clearpage
\subsection{Compute non-dimensionalizing constants}

Depending on the fields that are involved, there are 
currently four functions shown in 
Table~\ref{table:FunctionsNonDimensionalConstants}
which compute the non-dimensionalizing
constants given representative material parameters and geometry.
The key constants are, \ttt{M}(mass), \ttt{L}(length),
\ttt{T}(time), \ttt{A}(current), and \ttt{Th}(temperature),
which are essential in non-dimensionalizing the appropriate fields.
The auxiliary constants, \ttt{F}(force), \ttt{Qt}(thermal flux),
\ttt{V}(voltage), \ttt{Q}(charge), \ttt{E}(e-field), and
\ttt{R}(resistance), are derived from these key constants.
Their dimensions are presented in 
Table~\ref{table:DimensionsOfAuxiliaryVariables}.

\begin{table}[htbp]
\caption{Dimensions of auxiliary variables}
\label{table:DimensionsOfAuxiliaryVariables}
\vspace{0.1in}
\centering
\begin{tabular}{c|l|c|c|c|c|c}
\hline
\multicolumn{1}{c|}{\tbf{Variable name}} &
\multicolumn{1}{c|}{\tbf{Field name}} &
\ttt{M} & \ttt{L} & \ttt{T} & \ttt{Th} & \ttt{A} \\ 
\hline
\hline
\ttt{F} & force        &  1 &  1 & -2 & - & - \\
\hline
\ttt{Qt}& thermal flux &  1 &  - & -3 & - & - \\
\hline
\ttt{Q} & current      &  - &  - &  1 & - & 1 \\
\ttt{V} & voltage      &  1 &  2 & -3 & - &-1 \\
\ttt{E} & e-field      &  1 &  1 & -3 & - &-1 \\
\ttt{R} & resistance   &  1 &  2 & -3 & - &-2 \\
\hline
\end{tabular}
\end{table}

Further details on how these are computed from the material 
properties are presented below in the codelist.
\begin{codelist}
  \item[mech\_nondim(mtype,cL)]
    Computes characteristic scales (\ttt{M,L,T}) and dimensional 
    quantaties (\ttt{F}) from a 
    table of material properties \ttt{mtype} and characteristic length
    scale \ttt{cL}, which are used for non-dimensionalization in a 
    mechanical problem. \ttt{mtype} must have the fields 
    \ttt{rho}(mass density) and \ttt{E} (Youngs modulus).
    \begin{eqnarray}
      \text{M} &=& \text{rho} * \text{cL}^3 \nonumber \\ 
      \text{L} &=& \text{cL}                \nonumber \\
      \text{T} &=& \frac{\text{cL}}{\sqrt{\text{E/rho}}} \nonumber 
    \end{eqnarray}
    A table named \ttt{dim\_scales} with these fields is returned. 
  \item[ted\_nondim(mtype,cL)] 
    Computes characteristic scales (\ttt{M,L,T,Th}) and dimensional 
    quantaties (\ttt{F,Qt}) from a 
    table of material properties \ttt{mtype} and characteristic length
    scale \ttt{cL}, which are used for non-dimensionalization in a 
    mechanical problem. \ttt{mtype} must have the fields 
    \ttt{rho}(mass density) and \ttt{E} (Youngs modulus),
    \ttt{at}(linear coefficient of thermal expansion), 
    \ttt{cp}(specific heat at constant pressure),
    and \ttt{T0}(ambient temperature).
    \begin{eqnarray}
      \text{M} &=& \text{rho} * \text{cL}^3 \nonumber \\ 
      \text{L} &=& \text{cL}                \nonumber \\
      \text{T} &=& \frac{\text{cL}}{\sqrt{\text{E/rho}}} \nonumber \\
      \text{Th}&=& \frac{\text{T0 * at * E}}{\text{rho*cp}} \nonumber
    \end{eqnarray}
    A table named \ttt{dim\_scales} with these fields is returned.
  \item[pz\_nondim(mtype,cL)] 
    Computes characteristic scales (\ttt{M,L,T,A}) and dimensional 
    quantaties (\ttt{F,V,Q,E,R}) from a 
    table of material properties \ttt{mtype} and characteristic length
    scale \ttt{cL}, which are used for non-dimensionalization in a 
    piezoelectric mechanical problem. \ttt{mtype} must have the fields 
    \ttt{rho}(mass density) and \ttt{E} (Youngs modulus),
    \ttt{kds}(permitivity at constant stress), 
    and \ttt{d}(piezoelectric strain coefficients).
    \begin{eqnarray}
      \text{M} &=& \text{rho} * \text{cL}^3 \nonumber \\ 
      \text{L} &=& \text{cL}                \nonumber \\
      \text{T} &=& \frac{\text{cL}}{\sqrt{\text{E/rho}}} \nonumber \\
      \text{A} &=& \frac{\text{kds * cL}^2}{\text{d * T}} \nonumber 
    \end{eqnarray}
    A table named \ttt{dim\_scales} with these fields is returned.
  \item[em\_nondim(mtype,cL,(eps))] 
    Computes characteristic scales (\ttt{M,L,T,A}) and dimensional 
    quantaties (\ttt{F,V,Q,E,R}) from a 
    table of material properties \ttt{mtype} and characteristic length
    scale \ttt{cL}, which are used for non-dimensionalization in a 
    electromechanical problem. \ttt{mtype} must have the fields 
    \ttt{rho}(mass density) and \ttt{E} (Youngs modulus),
    and \ttt{eps}(permitivity at constant stress).
    \begin{eqnarray}
      \text{M} &=& \text{rho} * \text{cL}^3 \nonumber \\ 
      \text{L} &=& \text{cL}                \nonumber \\
      \text{T} &=& \frac{\text{cL}}{\sqrt{\text{E/rho}}} \nonumber \\
      \text{A} &=& \frac{\text{eps * E * cL}^3}{\text{T}} \nonumber
    \end{eqnarray}
\end{codelist}

\subsection{Non-dimensionalize material parameters}
\begin{codelist}
  \item[material\_normalize(ftype,...)]
    Non-dimensionalizes all arguments \ttt{...} by the the dimensional
    scale \ttt{dim\_scales[ftype]}. \ttt{ftype} must be a string corresponding
    to a characteristic scale (\ttt{M,L,T,Th,A}) or a dimensional quantaty
    \ttt{F,V,Q,E,R,Qt,etc.} that has been predefined in the table 
    \ttt{dim\_scales}. If such a field does not exist, 
    \ttt{dim\_scales[ftype]} is assumed one. The number of input arguments
    will be returned.
  \item[mech\_material\_normalize(m)]
    Non-dimensionalizes the mechanical fields of the material \ttt{m} 
    given as a table by the characteristic scales defined in \ttt{dim\_scales}, 
    and returns a table with ONLY these mechanical fields. There is no 
    restriction on the fields existing in \ttt{m}, but only the fields,
    \ttt{rho, E, lambda, mu, alpha, c11, c12, c13, c33, c55} will be 
    non-dimensionalized. 
    If a field in \ttt{dim\_scales} does not exist, it
    will be assumed one. 
  \item[ted\_material\_normalize(m)]
    Non-dimensionalizes the thermomechanical fields of the material \ttt{m} 
    given as a table by the characteristic scales defined in \ttt{dim\_scales}, 
    and returns a table with ONLY these thermomechanical fields. There is no 
    restriction on the fields existing in \ttt{m}, but only the fields,
    \ttt{rho, E, lambda, mu, alpha, at, cp, kt, T0} will be 
    non-dimensionalized.
    If a field in \ttt{dim\_scales} does not exist, it
    will be assumed one. 
  \item[pz\_material\_normalize(m)]
    Non-dimensionalizes the piezoelectric-mechanical fields of the 
    material \ttt{m} 
    given as a table by the characteristic scales defined in \ttt{dim\_scales}, 
    and returns a table with ONLY these piezoelectric-mechanical fields. 
    There is no 
    restriction on the fields existing in \ttt{m}, but only the fields,
    \ttt{rho, c11, c12, c13, c33, c55, d16, d31, d33, kds1, kds3} will be 
    non-dimensionalized.
    If a field in \ttt{dim\_scales} does not exist, it
    will be assumed one. 
\end{codelist}
