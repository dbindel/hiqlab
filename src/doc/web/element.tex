\section{Element interface}

The {\tt Element} class in HiQLab really refers to an element
type.  Individual elements in the mesh are not represented by
separate objects; rather, we use a ``flyweight'' pattern in which
characteristics of generic element types are stored inside an
element object, and characteristics of particular instances are
stored externally (in the mesh object).  This external information
is passed to the element methods as a pointer to the mesh and an
element identifier.  In order to do element-by-element
initialization and assembly tasks, then, we loop over each element
in the mesh and make calls of the form
{\tt mesh->etypes[eltid]->method(mesh, eltid, ...)}.

\subsection{Variable slots}

Assembling element contributions into a global system requires that
all elements share a common idea of what nodal variable number is
assigned to what type of variable.  The variable slot mechanism
provides a way to reconcile the global assignment of nodal variable
numbers with a local assignment specific to each element type.

For example, suppose we wanted to perform a simulation that
combined mechanical, electrical, and thermal calculations.  We
might decide at the global level that variables 0-2 at each node
correspond to displacements, variable 3 corresponds to
electrostatic potential, and variable 4 corresponds to temperature.
But a thermomechanical element would not know about electrostatic
potential; from the perspective of that element, perhaps variables
0-2 should correspond to displacement and variable 3 should
correspond to temperature.  In this case, the slot array for the
coupling element would look like {\tt \{0, 1, 2, 4\}}.

\subsection{Element declarations}

\begin{verbatim}

class Element {
public:
    Element(int nslots);
    virtual ~Element();

    /** Initialize element eltid of the mesh.  Allocate element vars. */
    virtual void initialize(Mesh* mesh, int eltid);

    /** Mark element IDs as active */
    virtual void assign_ids(Mesh* mesh, int eltid);

    /** Assemble structure */
    virtual void assemble_struct(Mesh* mesh, int eltid, QStructAssembler* K);

    /** Assemble element matrices */
    virtual void assemble_dR(Mesh* mesh, int eltid, QAssembler* K,
                             double cx=1, double cv=0, double ca=0);

    /** Assemble element residual */
    virtual void assemble_R(Mesh* mesh, int eltid);

    /** Compute the (lumped) L2 projection of a field defined
     *  at Gauss nodes.  Output is Mdiag += sum(integral N*N') and
     *  xfields += integral N*xfunc.
     */
    virtual void project_L2(Mesh* mesh, int eltid, int nfields,
                            double* Mdiag, double* xfields,
                            FieldEval& xfunc);

    /** Compute the stresses for an element at parent coordinates X.
     *  Returns an incremented stress pointer.  Use two calls to
     *  compute the stress associated with a complex displacement.
     */
    virtual double* stress(Mesh* mesh, int eltid, double* X, double* stress);

    /** Compute the time-averaged energy flux for an element at parent
     *  coordinates X.  Returns an incremented pointer.
     */
    virtual double* mean_power(Mesh* mesh, int eltid, double* X, double* EX);

    /** Read and write the id_slot map
     */
    int  num_id_slots()        { return id_slots.size();  }
    int  id_slot(int i) const  { return id_slots[i]; }
    int& id_slot(int i)        { return id_slots[i]; }
    void id_slot(int i, int x) { id_slots[i] = x;   }
    int  max_id_slot();

protected:
    std::vector<int> id_slots;    // id_slots[i] = slot for ith local var

    /** Default initializer -- just marks the mesh ID array. */
    void assign_ids_default(Mesh* mesh, int elt);

    void set_local_arrays(Mesh* mesh, int eltid,
                          int* id, double* nodex, int ndm);
    void set_local_id(Mesh* mesh, int eltid, int* id);
    static void set_local_x(Mesh* mesh, int eltid, double* nx, int ndm);

    void set_local_u (Mesh* mesh, int eltid, double*   nu);
    void set_local_ui(Mesh* mesh, int eltid, double*   nu);
    void set_local_u (Mesh* mesh, int eltid, dcomplex* nu);
    void set_local_v (Mesh* mesh, int eltid, double*   nv);
    void set_local_vi(Mesh* mesh, int eltid, double*   nv);
    void set_local_v (Mesh* mesh, int eltid, dcomplex* nv);
    void set_local_a (Mesh* mesh, int eltid, double*   na);
    void set_local_ai(Mesh* mesh, int eltid, double*   na);
    void set_local_a (Mesh* mesh, int eltid, dcomplex* na);

    void add_local_f (Mesh* mesh, int eltid, double*   nodef1);
    void add_local_f (Mesh* mesh, int eltid, dcomplex* nodef1);
};

\end{verbatim}
\subsection{Element construction and destruction}

The only internal data structure common to all element types is the
variable slot map.  By default, we initialize this map to an
appropriately-sized identity.  We defer to a higher-level routine
(typically in the Lua code) the coordination of how slots should be
assigned in a multiphysics problem

Because this is a base class, we need a virtual destructor.  Said
destructor doesn't need to do anything, though.

\begin{verbatim}

#define ME Element


ME::ME(int nslots) :
    id_slots(nslots)
{
    for (int i = 0; i < nslots; ++i)
        id_slots[i] = i;
}


ME::~ME()
{
}


\end{verbatim}
\subsection{Element initialization}

The element initialization phase is where the element tells
HiQLab how many branch variables and history variables it will
need.  It does this by setting {\tt mesh->nbranch(eltid)} and
{\tt mesh->nhist(eltid)}.  An element without branch or history
variables can use the default (empty) initialization routine.

\begin{verbatim}

void ME::initialize(Mesh* mesh, int eltid)
{
}


\end{verbatim}
\subsection{Allocating variables}

The {\tt assign\_ids} method marks the nodal and branch variables
that it will use.  By default, we assume that this includes all the
nodal variables listed in the slots array at each node attached to
the element, plus any branch variables that were allocated.  Some
element types might not use all variables at all nodes, though; for
example, a mixed pressure-displacement formulation could have
pressure only associated with an internal node.

\begin{verbatim}

void ME::assign_ids(Mesh* mesh, int eltid)
{
    assign_ids_default(mesh, eltid);
}


void ME::assign_ids_default(Mesh* mesh, int eltid)
{
    int nen = mesh->get_nen(eltid);
    for (int j = 0; j < nen; ++j) {
        int nodeid = mesh->ix(j,eltid);
        for (int i = 0; i < num_id_slots(); ++i)
            mesh->id(id_slots[i], nodeid) = 1;
    }
    int nbranch = mesh->nbranch_id(eltid);
    for (int j = 0; j < nbranch; ++j)
        mesh->branchid(j, eltid) = 1;
}


\end{verbatim}
\subsection{Assembling the residual and tangent}

The tangent stiffness looks like
\[
K = c_u \frac{\partial R}{\partial u} +
c_v \frac{\partial R}{\partial v} +
c_a \frac{\partial R}{\partial a}
\]
where $u$, $v$, and $a$ are the displacement, velocity, and
acceleration vectors, respectively.  The {\tt assemble\_dR} method
adds this element's contribution to $K$.  The
{\tt assemble\_struct} method just assembles the nonzero structure
of the tangent stiffness.

The residual vector $R(u,v,a)$ is assembled into the $F$ vector in
the mesh object.  {\tt assemble\_R} adds this element's
contribution.

\begin{verbatim}


void ME::assemble_struct(Mesh* mesh, int eltid, QStructAssembler* K)
{
    int nen = mesh->get_nen(eltid);
    int nslots = num_id_slots();
    int nbranch = mesh->nbranch_id(eltid);
    std::vector<int> id(nen*nslots+nbranch);
    set_local_id(mesh, eltid, &(id[0]));
    K->add(&(id[0]), nen*nslots+nbranch);
}


void ME::assemble_dR(Mesh* mesh, int eltid, QAssembler* K,
                     double cx, double cv, double ca)
{
}


void ME::assemble_R(Mesh* mesh, int eltid)
{
}


\end{verbatim}
\subsection{Lumped $L^2$ projections}

Computing a lumped $L^2$ projection of a quantity defined at the
Gauss points is a two-step procedure.  First, call the element
{\tt project\_L2} method on each element in the mesh.  The output
of this calculation should be
\begin{eqnarray*}
{\tt Mdiag(i)} & = & \int_{\Omega} N_i \, d\Omega \\
{\tt xfields(i,j)} & = & \int_{\Omega} N_i(x) v_j(x) \, d\Omega \\
\end{eqnarray*}
where $v_j(x)$ is the $j$th scalar component of the fields to be
evaluated, and $N_i(x)$ is the shape associated with a unit displcement
of variable $i$.

At the end of the initial loop, we scale {\tt xfields(i,j)} by
{\tt Mdiag(i)} in order to get the approximate $L^2$ projection of
the fields onto the nodal shape functions.
\begin{verbatim}


void ME::project_L2(Mesh* mesh, int eltid, int nfields,
                    double* Mdiag, double* xfields,
                    FieldEval& func)
{
}


\end{verbatim}
\subsection{Gauss point quantities}

Right now, there are two types of quantities that we can compute in
an element (typically at the Gauss points).  The first is the
stress components; the second is the mean energy flux in a
time-harmonic simulation.

The latter does seem fairly specific, and it's not coded for all
elements.  This piece of the infrastructure should probably be
handled more cleanly.

\begin{verbatim}


double* ME::stress(Mesh* mesh, int eltid, double* X, double* stress)
{
    return stress;
}


double* ME::mean_power(Mesh* mesh, int eltid, double* X, double* EX)
{
    return EX;
}


\end{verbatim}
\subsection{Counting the number of ID slots in use}

The {\tt max\_id\_slot} routine returns the last (global) variable
index listed in the slot array.  Taking the maximum of the
{\tt max\_id\_slot} return values over all elements tells us the
maximum number of variables we need to allocate per node.
I cannot think of any good reason that a user should call this
routine -- it only seems useful inside the mesh {\tt initialize}
method.

\begin{verbatim}


int ME::max_id_slot()
{
    int max_slot = 0;
    for (int i = 0; i < num_id_slots(); ++i)
        if (max_slot < id_slots[i])
            max_slot = id_slots[i];
    return max_slot;
}


\end{verbatim}
\subsection{Transferring data from global arrays}

For element calculations, we typically will want slices of the
global {\tt ID}, {\tt X}, {\tt U}, {\tt V}, and {\tt A}
arrays.  That's what the {\tt set\_local\_*} routines do.  Of these,
the only one that is less than obvious is the {\tt set\_local\_id}
function, which returns in its output array the list of all nodal
variables for the element, followed by the list of all branch
variables.

\begin{verbatim}


void ME::set_local_arrays(Mesh* mesh, int eltid,
                          int* id, double* nodex, int ndm)
{
    set_local_x(mesh,  eltid, nodex, ndm);
    set_local_id(mesh, eltid, id);
}


void ME::set_local_id(Mesh* mesh, int eltid, int* id1)
{
    int nen = mesh->get_nen(eltid);
    QMatrix<int> id(id1, num_id_slots(), nen);
    for (int j = 0; j < nen; ++j) {
        int nodeid = mesh->ix(j,eltid);
        for (int i = 0; i < num_id_slots(); ++i)
            id(i,j) = mesh->inode(id_slots[i], nodeid);
    }
    int nbranch = mesh->nbranch_id(eltid);
    for (int i = 0; i < nbranch; ++i)
        id1[nen*num_id_slots()+i] = mesh->ibranch(i,eltid);
}


void ME::set_local_x(Mesh* mesh, int eltid, double* nodex1, int ndm)
{
    int nen = mesh->get_nen(eltid);
    QMatrix<double> nodex(nodex1, ndm,nen);
    for (int j = 0; j < nen; ++j) {
        int nodeid = mesh->ix(j,eltid);
        for (int i = 0; i < ndm; ++i)
            nodex(i,j) = mesh->v(i,nodeid);
    }
}


#define getter1(name, t, v) \
void ME::name(Mesh* mesh, int eltid, t* nodex1)    \
{                                                       \
    int nen = mesh->get_nen(eltid);                     \
    QMatrix<t> nodex(nodex1, num_id_slots(), nen);      \
    for (int j = 0; j < nen; ++j) {                     \
        int nodeid = mesh->ix(j,eltid);                 \
        for (int i = 0; i < num_id_slots(); ++i)        \
            nodex(i,j) = mesh->v(id_slots[i],nodeid);   \
    }                                                   \
    int nbranch = mesh->nbranch_id(eltid);              \
    for (int i = 0; i < nbranch; ++i)                   \
        nodex1[nen*num_id_slots()+i] = mesh->v(mesh->ibranch(i,eltid));  \
}

getter1(set_local_u,  double,   u)
getter1(set_local_ui, double,   ui)
getter1(set_local_u,  dcomplex, uz)
getter1(set_local_v,  double,   v)
getter1(set_local_vi, double,   vi)
getter1(set_local_v,  dcomplex, vz)
getter1(set_local_a,  double,   a)
getter1(set_local_ai, double,   ai)
getter1(set_local_a,  dcomplex, az)


\end{verbatim}
\subsection{Assembling data from global arrays}

The output of some of the assembly loops goes to external storage,
but the {\tt assemble\_R} functions assemble their results into the
mesh {\tt F} array.  The {\tt add\_local\_f} functions take the
local element residual contribution and add it into this array in
the default way.

\begin{verbatim}


void ME::add_local_f(Mesh* mesh, int eltid, double* nodef1)
{
    int nen = mesh->get_nen(eltid);
    QMatrix<double> nodef(nodef1, num_id_slots(),nen);
    for (int j = 0; j < nen; ++j) {
        int nodeid = mesh->ix(j,eltid);
        for (int i = 0; i < num_id_slots(); ++i)
            mesh->f(id_slots[i],nodeid) += nodef(i,j);
    }
    int nbranch = mesh->nbranch_id(eltid);
    for (int i = 0; i < nbranch; ++i)
        mesh->f(mesh->ibranch(i,eltid)) += nodef1[nen*num_id_slots()+i];
}


void ME::add_local_f(Mesh* mesh, int eltid, dcomplex* nodef1)
{
    int nen = mesh->get_nen(eltid);
    QMatrix<dcomplex> nodef(nodef1, num_id_slots(),nen);
    for (int j = 0; j < nen; ++j) {
        int nodeid = mesh->ix(j,eltid);
        for (int i = 0; i < num_id_slots(); ++i) {
            mesh->f (id_slots[i],nodeid) += real(nodef(i,j));
            mesh->fi(id_slots[i],nodeid) += imag(nodef(i,j));
        }
    }
    int nbranch = mesh->nbranch_id(eltid);
    for (int i = 0; i < nbranch; ++i) {
        mesh->f (mesh->ibranch(i,eltid)) += real(nodef1[nen*num_id_slots()+i]);
        mesh->fi(mesh->ibranch(i,eltid)) += imag(nodef1[nen*num_id_slots()+i]);
    }
}
\end{verbatim}
