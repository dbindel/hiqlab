\section{Mesh objects}

The mesh data structure keeps track of the problem description and
the state of the system.  It does {\em not} keep track of the
assembled stiffness matrix, nor the intermediate vectors used in
Newton iterations, time-steps, etc; those are stored external to
the mesh.


\subsection{Mesh internal data structures}


\subsubsection{Node positions and connectivity}

We use a variation on the standard {\tt X} and {\tt IX} arrays for
representing the node positions and connectivities.  We require
that each node have {\tt ndm} coordinates (so {\tt X} is an 
{\tt ndm}-by-{\tt numnp} array), but we allow a variable number of nodes
per element.  In particular, we have in mind the possibility that
macro-elements connected to many nodes may be stored in the same
structure as ordinary elements connected to relatively few nodes.
Therefore, we represent the element connectivity using two arrays:
{\tt NIX} and {\tt IX}.  {\tt NIX} is an array of {\tt numelt + 1}
numbers such that element {\tt i} is connected to nodes {\tt IX[j]},
where {\tt NIX[i] <= j < NIX[i+1]}.

In addition to the element-to-node connectivity structure
represented by {\tt IX} and {\tt NIX}, there is also a
node-to-element connectivity data structure represented by {\tt IN}
and {\tt NIN}.  This reverse map currently is not used for very much.


\subsubsection{Element data}

Each element is associated with an element object which is
responsible for evaluating the local stiffness and residual,
marking which degrees of freedom it uses, etc.  These associations
are stored in the {\tt etypes} array.  In addition, there is an
array {\tt etypes\_owned} which is used to store what element
objects the mesh has ownership of (i.e. which elements should be
destroyed when the mesh object is destroyed).  Not all elements in
{\tt etypes\_owned} need be referenced from {\tt etypes}, and
vice-versa.


\subsubsection{Variable assignments}

There are three types of variables in HiQLab: {\em nodal} variables
associated with a particular node; {\em branch} variables
associated with a particular element; and {\em global} variables
which are associated with global shape functions.  Each of these
variables is associated with a position in the various global
arrays: all the nodal variables first, then all the branch
variables, then all the globals.  The index space is divvied up as
follows:
\begin{itemize}
\item {\tt i+j*maxndf}           -- node {\tt j}, variable {\tt i}
\item {\tt NB[j] <= k < NB[j+1]} -- element {\tt j} branch variables
\item {\tt NG[0] <= k < NG[1]}   -- global variables
\end{itemize}

The {\tt ID} array maps this full set of variable indices into a
reduced set of indices.  Only active variables are assigned valid
reduced indices; variables subject to displacement boundary
conditions are assigned negative indices.


\subsubsection{Boundary conditions}

Each variable in the system can be assigned a displacement boundary
condition ({\tt BC[k] = 'u'}), a force boundary condition 
({\tt BC[k] = 'f'}), or no boundary condition ({\tt BC[k] = ' '}).  For
those variables subject to boundary conditions, the corresponding
nodal boundary values are stored in the {\tt BV} and {\tt BVi} arrays.


\subsubsection{System state}

The {\tt U}, {\tt V}, and {\tt A} arrays contain the unreduced
displacement, velocity, and acceleration variables.  There are
parallel arrays {\tt Ui}, {\tt Vi}, and {\tt Ai} to contain the
imaginary parts.  The {\tt F} and {\tt Fi} variables contain the
force residual vector.

The force variables associated with global shape functions must be
computed in a separate pass after the ordinary (nodal and branch)
forces.  Right now, the code to deal with this is scattered
somewhat haphazardly through the mesh implementation.  A possibly
better alternative would be to introduce an explicit representation
of the global shape function matrix $G$, so that forming the full
residual would involve an ordinary sparse matvec and forming the
full tangent would involve an ordinary sparse matrix triple
product.


\subsubsection{History database}

HiQLab provides space for history data, though at present none of
the elements use it.  The history database consists of three
vectors.  The index vector {\tt NH} tracks which variables are
allocated to which element; element {\tt j} is assigned slots 
{\tt NH[j] <= k < NH[j+1]}.  The history data itself is stored in the
vectors {\tt HIST} and {\tt HISTi}.  The history vectors actually
each have storage for two sets of variables, one for reading and
one for writing.  Which set of variables represents the saved state
is represented by {\tt which\_hist} (which is 0 if the first half
of the database is the saved state, 1 if the second half is the
saved state).


\subsubsection{Scaling data}

HiQLab keeps two vectors of characteristic scales: {\tt D1}
represents the characteristic scales of the input vectors, and 
{\tt D2} represents the characteristic scales of the output vectors.
Therefore, if $\tilde{u}$, $\tilde{F}$, and $\tilde{K}$ are
displacements, forces, and stiffnesses in some system of units,
$u = D_1^{-1} \tilde{u}$, $F = D_2^{-1} \tilde{F}$, and $K =
D_2^{-1} \tilde{K} D_1$ represent nondimensionalized versions of
the displacements, forces, and stiffnesses.  By keeping this
scaling data in a single vector, we get the numerical advantages of
nondimensionalization (in terms of sensitivity of results, ease of
choosing convergence criteria, etc), but we can keep looking at
numbers with units when numbers with units make sense.


\subsubsection{Lua interpreter}

For a typical mesh object, we will have an associated Lua
interpreter which contains callback functions to set the boundary
conditions and scaling vectors in an appropriate manner.
The mesh may or may not also have ownership of the Lua interpreter.
If the mesh does have ownership of the interpreter, then the mesh
object destructor will close the interpreter.  Thus mesh objects
created from MATLAB or some other interface would typically have
ownership of their Lua interpreter, while mesh objects created from
the Lua interface might well use the Lua interpreter associated
with the environment in which they were created.
\subsection{Accessor functions}

The variables in our system can be real or complex, and they can be
associated with nodal variables, branch variables, or global variables.
Rather than write many tiny accessor functions by hand, we write a
few macros to produce these accessor functions.

\begin{verbatim}

#define defq_z_accessor1(name, array, index) \
   double&  name   (int i) { return array   [index]; } \
   double&  name##i(int i) { return array##i[index]; } \
   dcomplex name##z(int i) { return dcomplex(name(i), name##i(i));  }

#define defq_z_accessor2(name, array, index) \
   double&  name   (int i, int j) { return array   [index]; } \
   double&  name##i(int i, int j) { return array##i[index]; } \
   dcomplex name##z(int i, int j) { return dcomplex(name(i,j), name##i(i,j)); }

#define defq_z_accessor(name, array, index) \
   defq_z_accessor1(name, array, i) \
   defq_z_accessor2(name, array, index)

#define defq_meshz_accessor(u, U) \
   defq_z_accessor(u, U, j*maxndf+i) \
   defq_z_accessor(branch##u, U, NB[j]+i) \
   defq_z_accessor1(global##u, U, NG[0]+i)

#define defq_mesh_accessor(type, name, array) \
   type& name        (int i)        { return array[i];          } \
   type& name        (int i, int j) { return array[j*maxndf+i]; } \
   type& branch##name(int i, int j) { return array[NB[j]+i];    } \
   type& global##name(int i)        { return array[NG[0]+i];    }


\end{verbatim}
\subsection{Mesh declarations}
\begin{verbatim}

class Mesh {
 public:

    Mesh(int ndm, int maxnen = 0, int maxndf = 0);
    virtual ~Mesh();

    /** Get Lua state used by (not necessarily owned by) system */
    lua_State* get_lua()          { return L; }
    void set_lua(lua_State* argL) { L = argL; }

    /** Get out a named scale from the Lua state */
    double get_scale(const char* name);

    /** Set scaling vector D1 and D2 */
    void clear_scaling_vector();
    void set_nodal_u_scale(int i, double s);
    void set_nodal_f_scale(int i, double s);
    void set_scaling_vector();

    /** Get scaling vector D1 or D2 */
    void get_scaling_vector(double* su, double* sf, int is_reduced = 1);

    /** Add node(s), element(s), or global(s) to the mesh directly.
     *  In each case, return the identifier of the first item created.
     */
    int add_node(double* x, int n=1);
    int add_element(int* e, Element* etype, int nen, int n=1);
    int add_global(int n);

    /** Construct a Cartesian block. */
    void add_block(double* x1, double* x2, int* m,
                   Element* etype, int order=1);
    void add_block(double x1, double x2, int m,
                   Element* etype, int order=1);
    void add_block(double x1, double y1,
                   double x2, double y2,
                   int nx, int ny,
                   Element* etype, int order=1);
    void add_block(double x1, double y1, double z1,
                   double x2, double y2, double z2,
                   int nx, int ny, int nz,
                   Element* etype, int order=1);

    /** Tie mesh nodes in a range.  Only affects the element connectivity
     *  array; tied nodes are not removed from the mesh data structure.
     */
    void tie(double tol, int start=-1, int end=-1);

    /** Remove any nodes which are not referenced by the IX array.
     *  Should be called immediately after tie().
     */
    void remove_unused_nodes();

    /** Get a clean mesh which has no redundant nodes, i.e. mesh
     *  which has no nodes with dof labeled NULL(-1)
     */
    Mesh* get_clean_mesh();

    /** Allocate space for global arrays and assign initial index map. */
    void initialize();

    /** Reassign the index map. */
    int assign_ids();

    /** Assemble tangent matrix structure. */
    void assemble_struct(QStructAssembler* K);

    /** Assemble global mass and stiffness matrices. */
    void assemble_dR(QAssembler* K, double cx=1, double cv=0, double ca=0);

    /** Assemble global reaction vector (goes into f) */
    void assemble_R();

    /** Assemble time-averaged energy flux at node points */
    void mean_power(double* E);

    /** Clear current boundary condition */
    void clear_bc();

    /** Use Lua to set boundary conditions */
    void set_bc         (const char* funcname);
    void set_globals_bc (const char* funcname);
    void set_elements_bc(const char* funcname);

    /** Apply boundary conditions based on saved Lua functions */
    void apply_bc();

    /** Build alternate vectors (for drive or sense) using Lua */
    void get_vector(const char* fname, double* v, int is_reduced = 1);

    /** Evaluate a Lua function at every nodal point */
    void get_lua_fields(const char* funcname, int n, double* fout);

    /** Evaluate global shape functions via Lua funcs */
    double shapeg(int i, int j);
    void   shapeg(double* v, double c, int j,
                  int is_reduced, int vstride = 1);

    /** Copy i*omega*u and -omega^2*u into a and v */
    void make_harmonic(dcomplex omega);
    void make_harmonic(double omega_r, double omega_i = 0) {
        make_harmonic(dcomplex(omega_r, omega_i));
    }

    int numnp()         { return X.size()  / ndm;    }
    int numelt()        { return etypes.size();      }
    int numglobals()    { return NG[1]-NG[0];        }
    int get_numid()     { return numid;              }
    int get_ndm()       { return ndm;                }
    int get_ndf()       { return maxndf;             }
    int get_nen()       { return maxnen;             }
    int nbranch_id()    { return NB[numelt()]-NB[0]; }

    int       get_nen    (int i) { return NIX[i+1]-NIX[i]; }
    int       get_nne    (int i) { return NIN[i+1]-NIN[i]; }
    int       nbranch_id (int i) { return NB[i+1] -NB[i];  }
    int       nhist_id   (int i) { return NH[i+1] -NH[i];  }
    Element*& etype      (int i) { return etypes[i];       }
    int&      nbranch    (int i) { return NB[i];           }
    int&      nhist      (int i) { return NH[i];           }

    int&      ix(int i, int j) { return IX[NIX[j] + i]; }
    int&      in(int i, int j) { return IN[NIN[j] + i]; }
    double&   x (int i, int j) { return X[j*ndm + i];   }

    /** Get the global array indices for different variables */
    int inode  (int i, int j) { return j*maxndf+i; }
    int ibranch(int i, int j) { return NB[j]+i;    }
    int iglobal(int i)        { return NG[0]+i;    }

    /** Get and set the ID and BC arrays */
    defq_mesh_accessor( int,  id, ID )
    defq_mesh_accessor( char, bc, BC )

    /** Get and set the boundary values, u, v, a, and f arrays */
    defq_meshz_accessor( bv,  BV  )
    defq_meshz_accessor( u ,  U   )
    defq_meshz_accessor( v ,  V   )
    defq_meshz_accessor( a ,  A   )
    defq_meshz_accessor( f ,  F   )

    void set_u (double* u = NULL, double* v = NULL, double* a = NULL);
    void set_ui(double* u = NULL, double* v = NULL, double* a = NULL);
    void set_uz(double* u = NULL, double* ui = NULL,
                double* v = NULL, double* vi = NULL,
                double* a = NULL, double* ai = NULL);
    void set_uz(dcomplex* u = NULL, dcomplex* v = NULL, dcomplex* a = NULL);

    void set_f (double* f = NULL);
    void set_fi(double* f = NULL);
    void set_fz(double* fr = NULL, double* fi = NULL);
    void set_fz(dcomplex* f = NULL);

    void get_u (double* u = NULL, double* v = NULL, double* a = NULL);
    void get_ui(double* u = NULL, double* v = NULL, double* a = NULL);
    void get_uz(double* u = NULL, double* ui = NULL,
                double* v = NULL, double* vi = NULL,
                double* a = NULL, double* ai = NULL);
    void get_uz(dcomplex* u = NULL, dcomplex* v = NULL, dcomplex* a = NULL);

    void get_f (double* f = NULL);
    void get_fi(double* f = NULL);
    void get_fz(double* fr = NULL, double* fi = NULL);
    void get_fz(dcomplex* f = NULL);

    /** Access element history entries */
    double get_hist (int i, int j) { return HIST [hist_index(i,j,0)]; }
    double get_histi(int i, int j) { return HISTi[hist_index(i,j,0)]; }
    double put_hist (int i, int j, double x) { HIST [hist_index(i,j,1)] = x; }
    double put_histi(int i, int j, double x) { HISTi[hist_index(i,j,1)] = x; }
    void swap_hist() {
        which_hist = (which_hist+1)%2;
    }

    /** Get and set scaling parameters for primary and secondary variables */
    defq_mesh_accessor( double, d1,  D1  )
    defq_mesh_accessor( double, d2,  D2  )

    Element* own(Element* e);
    void     own(lua_State* L);

    void lua_getfield(const char* name);
    void lua_setfield(const char* name);
    int  lua_method(const char* name, int nargin, int nargout);

\end{verbatim}
\begin{verbatim}
};

\end{verbatim}
\subsection{Mesh construction, destruction, and memory management}

Most of the data in the mesh object lives in C++ containers, which
are automatically managed.  An exception to this is the element types
and the Lua state.  We require that the user explicitly hand over
ownership of these types for two reasons.  First, it may not always
make sense for the mesh object to own these objects -- for example,
when using the Lua front-end, the Lua interpreter will typically outlive
the mesh!  Second, the mesh object isn't responsible for constructing
these objects, so unless there is some explicit registration mechanism,
the mesh will not necessarily know that the objects exist.

It is a checked error to try to unassign or reassign ownership of
the Lua state after the first assignment.

\begin{verbatim}


Mesh::Mesh(int ndm, int maxnen, int maxndf) :
    ndm(ndm), maxnen(maxnen), maxndf(maxndf), numid(0), which_hist(0),
    lua_owned(NULL), L(NULL)
{
    NG[0] = 0;
    NG[1] = 0;
    NIX.push_back(0);
}


Mesh::~Mesh()
{
    for (unsigned i = 0; i < etypes_owned.size(); ++i)
        if (etypes_owned[i])
            delete etypes_owned[i];
    if (lua_owned)
        lua_close(lua_owned);
}


Element* Mesh::own(Element* e)
{
    etypes_owned.push_back(e);
    return e;
}


void Mesh::own(lua_State* L)
{
    assert(lua_owned == NULL);
    assert(L != NULL);
    lua_owned = L;
}


\end{verbatim}
\subsection{Mesh scale management}

We keep two distinct sets of scaling information associated with a
mesh.  

First, there is the abstract scaling information associated with
different variable types (temperatures, displacements, etc).  The
user provides access to these scales by associating a
{\tt get\_scale} method with the Lua mesh object.  This method maps
a string argument to a numeric scaling value.  If the method isn't
there, or doesn't return anything, we return a default scale value
of 1.

Second, there are scales associated with each primary and dual
variable in the system (including element and global variables).
These scales are stored in the {\tt d1} and {\tt d2} arrays,
respectively.  The default scales are all ones, but the defaults
would typically be overridden by the {\tt set\_scaling\_vector}
method.  This method in turn calls {\tt set\_nodal\_u\_scale} and
{\tt set\_nodal\_f\_scale} to set all the scales associated with a
particular slot in the nodal variable array (since a given slot is
associated with a particular field -- see the element interface
description).

\begin{verbatim}


double Mesh::get_scale(const char* name)
{
    CHECK_LUA 1;
    lua_pushstring(L, name);
    double retval = 1;
    if (lua_method("get_scale", 1, 1) == 0) {
        retval = lua_tonumber(L,-1);
        lua_pop(L,1);
    }
    return retval;
}


void Mesh::clear_scaling_vector()
{
    int N = ID.size();
    for (int i = 0; i < N; ++i) {
        d1(i) = 1;
        d2(i) = 1;
    }
}


void Mesh::set_nodal_u_scale(int i, double s)
{
    for (int j = 0; j < numnp(); ++j)
        d1(i,j) = s;
}


void Mesh::set_nodal_f_scale(int i, double s)
{
    for (int j = 0; j < numnp(); ++j)
        d2(i,j) = s;
}


void Mesh::set_scaling_vector()
{
    CHECK_LUA;
    lua_method("set_scaling_vector", 0, 0);
}


void Mesh::get_scaling_vector(double* su, double* sf, int is_reduced)
{
    int N = (is_reduced ? ID.size() : numnp()*maxndf );
    for (unsigned i = 0; i < (unsigned) N; ++i) {
        int k = (is_reduced ? id(i) : i);
        if (k >= 0) {
            if (su) su[k] = D1[i];
            if (sf) sf[k] = D2[i];
        }
    }
}


\end{verbatim}
\subsection{Basic mesh generation}

\begin{verbatim}


int Mesh::add_node(double* x, int n)
{
    int id = X.size() / ndm;
    for (int i = 0; i < n*ndm; ++i)
        X.push_back(x[i]);
    return id;
}


int Mesh::add_global(int n)
{
    int id = NG[1];
    NG[1] += n;
    return id;
}


int Mesh::add_element(int* e, Element* etype, int nen, int n)
{
    int id = etypes.size();
    for (int i = 0; i < n; ++i) {
        for (int j = 0; j < nen; ++j)
            IX.push_back(e[i*nen+j]);
        end_add_element(etype);
    }
    return id;
}


void Mesh::end_add_element(Element* etype)
{
    NIX.push_back(IX.size());
    etypes.push_back(etype);
    NB.push_back(0);
    NH.push_back(0);
    if (etype)
        etype->initialize(this, etypes.size()-1);
}


\end{verbatim}
\subsection{Block mesh generation}

The block mesh generators produce rectilinear coordinate-aligned
block meshes, which can subsequently be turned into more
interesting shape by mapping and tying operations.  For each
dimension, we specify the coordinate range ($[x_1, x_2]$) and the
number of {\em nodes} (not elements) in that direction.  It is a
checked error to try to specify a number of nodes that doesn't
yield an integer number of elements.

\begin{verbatim}


void Mesh::add_block(double* x1, double* x2, int* m,
                     Element* etype, int order)
{
    if (ndm == 1)
        add_block(x1[0], x1[0], m[0], etype, order);

    else if (ndm == 2)
        add_block(x1[0], x1[1],
                  x2[0], x2[1],
                  m[0],  m[1], etype, order);

    else if (ndm == 3)
        add_block(x1[0], x1[1], x1[2],
                  x2[0], x2[1], x2[2],
                  m[0],  m[1],  m[2], etype, order);
}


void Mesh::add_block(double x1, double x2, int nx,
                     Element* etype, int order)
{
    assert( nx > 0 && (nx-1)%order == 0 );

    int start_node = numnp();
    for (int ix = 0; ix < nx; ++ix) {
        double x[1];
        x[0] = ((nx-1-ix)*x1 + ix*x2) / (nx-1);
        add_node(x);
    }

    for (int ix = 0; ix < nx-order; ix += order) {
        for (int jx = 0; jx < order+1; ++jx)
            IX.push_back( (ix+jx) + start_node );
        end_add_element(etype);
    }
}


void Mesh::add_block(double x1, double y1,
                     double x2, double y2,
                     int nx, int ny,
                     Element* etype, int order)
{
    assert( nx > 0 && (nx-1)%order == 0 );
    assert( ny > 0 && (ny-1)%order == 0 );

    int start_node = numnp();
    for (int ix = 0; ix < nx; ++ix) {
        for (int iy = 0; iy < ny; ++iy) {
            double x[2];
            x[0] = ((nx-1-ix)*x1 + ix*x2) / (nx-1);
            x[1] = ((ny-1-iy)*y1 + iy*y2) / (ny-1);
            add_node(x);
        }
    }

    for (int ix = 0; ix < nx-order; ix += order)
        for (int iy = 0; iy < ny-order; iy += order) {
            for (int jx = 0; jx < order+1; ++jx)
                for (int jy = 0; jy < order+1; ++jy)
                    IX.push_back( (iy+jy) + ny*(ix+jx) + start_node );
            end_add_element(etype);
        }
}


void Mesh::add_block(double x1, double y1, double z1,
                     double x2, double y2, double z2,
                     int nx, int ny, int nz,
                     Element* etype, int order)
{
    assert( nx > 0 && (nx-1)%order == 0 );
    assert( ny > 0 && (ny-1)%order == 0 );
    assert( nz > 0 && (nz-1)%order == 0 );

    int start_node = numnp();

    for (int ix = 0; ix < nx; ++ix) {
        for (int iy = 0; iy < ny; ++iy) {
            for (int iz = 0; iz < nz; ++iz) {
                double x[3];
                x[0] = ((nx-1-ix)*x1 + ix*x2) / (nx-1);
                x[1] = ((ny-1-iy)*y1 + iy*y2) / (ny-1);
                x[2] = ((nz-1-iz)*z1 + iz*z2) / (nz-1);
                add_node(x);
            }
        }
    }

    for (int ix = 0; ix < nx-order; ix += order)
        for (int iy = 0; iy < ny-order; iy += order)
            for (int iz = 0; iz < nz-order; iz += order) {
                for (int jx = 0; jx < order+1; ++jx)
                    for (int jy = 0; jy < order+1; ++jy)
                        for (int jz = 0; jz < order+1; ++jz)
                            IX.push_back( (iz+jz) + nz*((iy+jy) + ny*(ix+jx)) +
                                          start_node );
                end_add_element(etype);
            }
}


\end{verbatim}
\subsection{Tied meshes}

We tie meshes together by approximately sorting the coordinates,
looking first at $x$, then $y$, then $z$.  Nodes that are
equivalent under this approximate sort (and therefore appear
together in the sorted node list) are tied.  For each equivalence
class, we pick one representative node, and map all references to
other nodes in the class into references to that representative.

The assumption in this algorithm is that all nodes that are
supposed to be tied together are within a given tolerance of each
other (and all nodes that are not supposed to be tied are farther
than that tolerance).  It's possible to build a pathological mesh
where this assumption fails; for example, if we have nodes at
$(0,0)$, $(0,\epsilon)$, $(0,-\epsilon)$ where $\epsilon$ is the
tolerance, then the relation we've defined is not transitive, and
we don't have an equivalence relation.  In this case, things that
should probably be tied together might not be.  I regard this as
an unchecked blunder on the part of the analyst.

\begin{verbatim}


class CoordSorter {
public:
    CoordSorter(int ndm, Mesh& mesh, double tol) :
        ndm(ndm), mesh(mesh), tol(tol) {}

    int operator()(int i, int j)
    {
        for (int k = 0; k < ndm; ++k) {
            if (mesh.x(k,i) < mesh.x(k,j)-tol)   return  1;
            if (mesh.x(k,i) > mesh.x(k,j)+tol)   return  0;
        }
        return 0;
    }

    int compare(int i, int j)
    {
        for (int k = 0; k < ndm; ++k) {
            if (mesh.x(k,i) < mesh.x(k,j)-tol)   return  -1;
            if (mesh.x(k,i) > mesh.x(k,j)+tol)   return   1;
        }
        return 0;
    }

private:
    int ndm;
    Mesh& mesh;
    double tol;
};


void Mesh::tie(double tol, int start, int end)
{
    int np = numnp();
    if (start < 0) start = 0;
    if (end   < 0) end   = np;
    int nt = end-start;

    // Set scratch = start:end-1, then sort the indices according
    // to the order of the corresponding nodes.
    //
    CoordSorter sorter(ndm, *this, tol);
    vector<int> scratch(nt);
    for (int i = 0; i < nt; ++i) 
        scratch[i] = start+i;
    sort(scratch.begin(), scratch.end(), sorter);

    // map[i] := smallest index belonging to equivalence class of p[i]
    //
    vector<int> map(np);
    for (int i = 0; i < np; ++i) 
        map[i] = i;

    int inext;
    for (int i = 0; i < nt; i = inext) {
        int class_id = scratch[i];
        for (inext = i+1; inext < nt; ++inext) {
             if (sorter.compare(scratch[i], scratch[inext]) != 0)
                 break;
             if (scratch[inext] < scratch[i])
                class_id = scratch[inext];
        }
        for (int j = i; j < inext; ++j)
            map[scratch[j]] = class_id;
    }

    // Apply map to element connectivity array
    //
    for (int i = 0; i < (int) IX.size(); ++i)
        if (IX[i] >= 0)
            IX[i] = map[IX[i]];
}


\end{verbatim}
\subsection{Removing redundant nodes}

{\tt get\_clean\_mesh} creates a new mesh in which any nodes that have
no assigned variables are removed.  The resulting mesh does not own any
storage; it is not initialized; and it does not have access to the
Lua interpreter.

{\tt remove\_unused\_nodes} removes all unused nodes from the mesh,
which I think is the functionality we actually wanted.  This routine
should be called before initialization (or the mesh should be
reinitialized immediately afterward).

\begin{verbatim}


Mesh* Mesh::get_clean_mesh()
{
    // -- Clear boundary conditions
    clear_bc();
    assign_ids();

    // -- Construct new mesh
    Mesh* mesh_n = new Mesh(ndm,maxnen,maxndf);

    // -- Add new nodes
    int numnp_n = 0;
    vector<double> x_n(X.size());
    vector<int>    map_n(numnp());
    for (int i = 0; i < numnp(); ++i) {
        int ifix = 0;
        for (int j = 0; j < maxndf; ++j)
            ifix += ID[maxndf*i+j];
        if (ifix > -maxndf) {
            map_n[i] = numnp_n;
            for (int k = 0; k < ndm; ++k)
                x_n[numnp_n*ndm+k] = X[i*ndm+k];
            numnp_n++;
        } else
            map_n[i] = -1;
    }
    mesh_n->add_node(&(x_n[0]), numnp_n);

    // -- Add element connectivity
    for (int i = 0; i < numelt(); ++i) {
        int nen = get_nen(i);
        int e_n[nen];
        for (int j = 0; j < nen; ++j)
            e_n[j] = map_n[ix(j,i)];
        mesh_n->add_element(e_n, etypes[i], nen, 1);
    }

    // -- Add globals
    mesh_n->add_global(numglobals());

    // -- Reinitialize current mesh
    apply_bc();
    set_scaling_vector();

    return mesh_n;
}


void Mesh::remove_unused_nodes()
{
    vector<int> map_used(numnp());
    clear(map_used);

    // Determine which nodes are used by IX
    for (int i = 0; i < IX.size(); ++i)
        map_used[IX[i]] = 1;

    // Compress out unused nodes, building an index map as we go
    int num_used = 0;
    for (int i = 0; i < numnp(); ++i) {
        if (map_used[i]) {
            for (int j = 0; j < ndm; ++j)
                x(num_used,i) = x(j,i);
            map_used[i] = num_used++;
        }
    }
    X.resize(ndm*num_used);

    // Remap the IX array
    for (int i = 0; i < IX.size(); ++i)
        IX[i] = map_used[IX[i]];
}


\end{verbatim}
\subsection{Initialization}

Mesh initialization occurs after the geometry of the mesh has been
established and before any analysis takes place.  This is where we
decide how big all the major arrays will be, and set up the index
structures that are used for all of our assembly operations.

\begin{verbatim}


void Mesh::initialize()
{
    build_reverse_map();              // Build node-to-element map
    initialize_sizes();               // Figure out maxnen and maxndf
    int M  = initialize_indexing();   // Assign branch and global dofs
    int MH = initialize_history();    // Assign history storage
    initialize_arrays(M, MH);         // Allocate storage arrays

    // -- Assign IDs, BCs, and scales
    assign_ids();
    apply_bc();
    set_scaling_vector();
}


void Mesh::initialize_sizes()
{
    for (int i = 0; i < etypes.size(); ++i) {
        int nen = get_nen(i);
        if (maxnen < nen)
            maxnen = nen;
    }
    for (int i = 0; i < etypes.size(); ++i) {
        if (etypes[i]) {
            int elt_maxndf = etypes[i]->max_id_slot() + 1;
            if (maxndf < elt_maxndf)
                maxndf = elt_maxndf;
        }
    }
}


void Mesh::build_reverse_map()
{
    int N = numnp();
    int M = etypes.size();
    clear(NIN,N+1);

    // -- NIN[i] := elements for node i
    for (unsigned i = 0; i < IX.size(); ++i)
        if (IX[i] >= 0)
            NIN[IX[i]]++;

    // -- NIN[i] := elements for nodes 0 .. i
    for (int i = 1; i < N+1; ++i)
        NIN[i] += NIN[i-1];

    // -- Build reverse map via bucket sort
    clear(IN, NIN[N]);
    for (int i = 0; i < M; ++i) {
        for (int j = NIX[i]; j < NIX[i+1]; ++j) {
            int node = IX[j];
            int slot = --NIN[node];
            IN[slot] = i;
        }
    }
}


int Mesh::initialize_indexing()
{
    // -- Assign index ranges for branch dofs
    int M = numnp()*maxndf;
    for (int i = 0; i < etypes.size(); ++i) {
        int start_M = M;
        M += NB[i];
        NB[i] = start_M;
    }
    NB.push_back(M);

    // -- Assign index ranges for global dofs
    NG[0] += M;
    NG[1] += M;
    M = NG[1];

    return M;
}


int Mesh::initialize_history()
{
    int MH = 0;
    for (int i = 0; i < etypes.size(); ++i) {
        int start_MH = MH;
        MH += NH[i];
        NH[i] = start_MH;
    }
    NH.push_back(MH);
    return MH;
}


void Mesh::initialize_arrays(int M, int MH)
{
    clear(U, Ui, M);
    clear(V, Vi, M);
    clear(A, Ai, M);
    clear(F, Fi, M);
    clear(BV,BVi,M);
    clear(BC,M);
    clear(ID,M);
    clear(D1,M);
    clear(D2,M);
    clear(HIST, HISTi, 2*MH);
}


\end{verbatim}
\subsection{Assigning reduced indices}

The {\tt assign\_ids} method sets up the map from the full index
set to the reduced index set.  Only the active variables (variables
that someone cares about that are not subject to a displacement BC)
are assigned valid reduced indices.  This method is called at
initialization, but it can also be called at subsequent phases in
the analysis if the user wishes to change the boundary conditions.

\begin{verbatim}


int Mesh::assign_ids()
{
    int N = etypes.size();   // Number of elements
    int M = ID.size();       // Number of potential dofs

    // -- Mark active dofs
    clear(ID);
    for (int i = 0; i < N; ++i) {
        if (etypes[i])
            etypes[i]->assign_ids(this, i);
    }
    for (int i = NG[0]; i < NG[1]; ++i)
        ID[i] = 1;

    // -- Assign IDs to nodal dofs
    numid = 0;
    for (int i = 0; i < M; ++i) {
        if (ID[i] != 0 && BC[i] != 'u')
            ID[i] = numid++;
        else
            ID[i] = -1;
    }

    return numid;
}


\end{verbatim}
\subsection{Assembly loops}

The {\tt assemble\_dR} routine assembles a linear combination of
mass, damping, and stiffness matrices.  The {\tt assemble\_struct}
routine is used to accumulate the nonzero structure of the matrix
for pre-allocation. 

The {\tt assemble\_R} routine assembles the residual.  It's a
little quirky because it not only assembles the residual
contributions from the elements, but it also computes the
components of the extended residual that are associated with global
shape functions.  That is, we compute something like
\[
R_g = G^T R_0
\]
where $R_g$ is the residual associated with the global shapes
described by the columns of $G$, and $R_0$ is the unreduced
residual without the global shape functions.

The {\tt mean\_power} loop assembles a lumped $L^2$ projection
of the mean power as computed at each Gauss point.  It's a sort of
quirky routine -- why just the mean power, and not also other Gauss
point quantities like the stress?  We should probably rethink this
routine t some point.

\begin{verbatim}


void Mesh::assemble_struct(QStructAssembler* K)
{
    for (unsigned i = 0; i < etypes.size(); ++i)
        if (etypes[i])
            etypes[i]->assemble_struct(this, i, K);
}


void Mesh::assemble_dR(QAssembler* K, double cx, double cv, double ca)
{
    for (unsigned i = 0; i < etypes.size(); ++i)
        if (etypes[i])
            etypes[i]->assemble_dR(this, i, K, cx, cv, ca);
}


void Mesh::assemble_R()
{
    set_fz((dcomplex*) NULL);

    for (unsigned i = 0; i < etypes.size(); ++i)
        if (etypes[i])
            etypes[i]->assemble_R(this, i);

    if (numglobals() > 0) {
        for (int j = 0; j < numglobals(); ++j) {
            vector<double> scratch;
            scratch.resize(ID.size());
            shapeg(&(scratch[0]), 1, j, 0);
            double fgjr = 0;
            double fgji = 0;
            for (unsigned k = 0; k < ID.size(); ++k) {
                fgjr += F [k]*scratch[k];
                fgji += Fi[k]*scratch[k];
            }
            globalf (j) += fgjr;
            globalfi(j) += fgji;
        }
    }
}


void Mesh::mean_power(double* E)
{
    FieldEvalPower mean_power_func;
    vector<double> Mdiag(numnp());
    memset(&(Mdiag[0]), 0, numnp()     * sizeof(double));
    memset(E,           0, numnp()*ndm * sizeof(double));
    for (unsigned i = 0; i < etypes.size(); ++i)
        if (etypes[i])
            etypes[i]->project_L2(this, i, ndm, &(Mdiag[0]), 
                                  E, mean_power_func);
    for (int i = 0; i < numnp(); ++i)
        if (Mdiag[i])
            for (int j = 0; j < ndm; ++j)
                E[i*ndm+j] /= Mdiag[i];
}


\end{verbatim}
\subsection{Boundary condition manipulations}

The boundary conditions are specified through the Lua fields 
{\tt bcfunc}, {\tt shapegbc}, and {\tt elementsbc}.  This is a little
bit of a relic, though, as all the actual work of setting up the
boundary conditions is now done through the {\tt form\_bcs}
function.  Since nothing in the C++ code cares directly about
anything but {\tt form\_bcs}, why do we still have C++ setters for
these other functions?

\begin{verbatim}


void Mesh::set_bc(const char* funcname)
{
    lua_getglobal(L, funcname);
    lua_setfield("bcfunc");
}


void Mesh::set_globals_bc(const char* funcname)
{
    lua_getglobal(L, funcname);
    lua_setfield("shapegbc");
}


void Mesh::set_elements_bc(const char* funcname)
{
    lua_getglobal(L, funcname);
    lua_setfield("elementsbc");
}


void Mesh::apply_bc()
{
    apply_lua_bc();

    // Reassign IDs
    assign_ids();

    // Copy boundary data into place
    set_bc_u(U, BV );
    set_bc_u(Ui,BVi);
    set_bc_f(F, BV );
    set_bc_f(Fi,BVi);

    // Update node disps that are linked to globals
    for (int j = 0; j < numglobals(); ++j) {
        shapeg(&(U [0]), U [NG[0]+j], j, 0);
        shapeg(&(Ui[0]), Ui[NG[0]+j], j, 0);
    }
}


void Mesh::apply_lua_bc()
{
    CHECK_LUA;
    lua_method("form_bcs", 0, 0);
}


void Mesh::clear_bc()
{
    clear(U,  Ui);
    clear(V,  Vi);
    clear(A,  Ai);
    clear(F,  Fi);
    clear(BV, BVi);
    clear(BC);
    clear(ID);
}


\end{verbatim}
\subsection{Setting up time-harmonic analysis}

For a time-harmonic motion with frequency $\omega$, we have
$\hat{v} = i \omega \hat{u}$ and $\hat{a} = i \omega \hat{v}$.
The {\tt make\_harmonic} motion sets $u$ and $a$ based on these
relations.

\begin{verbatim}


void Mesh::make_harmonic(dcomplex omega)
{
    dcomplex iomega = dcomplex(0,1)*omega;
    for (unsigned i = 0; i < U.size(); ++i) {
        dcomplex uu(U[i], Ui[i]);
        dcomplex vv = uu*iomega;
        dcomplex aa = vv*iomega;

        V [i] = real(vv);
        Vi[i] = imag(vv);
        A [i] = real(aa);
        Ai[i] = imag(aa);
    }
}


\end{verbatim}
\subsection{Lua accessors}

We use Lua functions to define the drive and sense vectors used in
transfer function evaluation, auxiliary fields that we might want
to plot, and the global shape functions.

\begin{verbatim}


void Mesh::get_vector(const char* fname, double* v, int is_reduced)
{
    CHECK_LUA;
    QVecAssembler va(v, NULL, this, is_reduced);
    lua_getglobal(L, fname);
    tolua_pushusertype(L, this, "Mesh");
    tolua_pushusertype(L, &va, "QVecAssembler");
    lua_pcall2(L, 2, 0);
}


void Mesh::get_lua_fields(const char* funcname, int n, double* fout)
{
    lua_pushstring(L, funcname);
    lua_gettable(L, LUA_GLOBALSINDEX);
    if (!lua_isfunction(L,-1)) {
        printf("Error in get_lua_fields: [%s]\n", funcname);
        lua_pop(L,1);
        return;
    }

    int func = lua_gettop(L);
    int N    = numnp();
    memset(fout, 0, n*N * sizeof(double));

    for (int j = 0; j < N; ++j) {
        int t = lua_gettop(L)+1;
        lua_pushvalue(L, func);
        for (int i = 0; i < ndm; ++i)
            lua_pushnumber(L, x(i,j));
        if (lua_pcall2(L, ndm, n) == 0) {
            for (int i = 0; i < n && i+t < (int) lua_gettop(L)+1; ++i)
                fout[j*n+i] = lua_tonumber(L, i+t);
        }
        lua_pop(L, n);
    }
    lua_pop(L,1);
}


double Mesh::shapeg(int i, int j)
{
    CHECK_LUA 0;
    lua_pushnumber(L, i);
    lua_pushnumber(L, j);
    double retval = 0;
    if (lua_method("get_shapeg", 2, 1) == 0) {
        retval = lua_tonumber(L,-1);
        lua_pop(L,1);
    }
    return retval;
}


void Mesh::shapeg(double* v, double c, int j, int is_reduced, int vstride)
{
    if (!L || c == 0)
        return;

    QVecAssembler va(v, NULL, this, is_reduced, vstride);
    tolua_pushusertype(L, &va, "QVecAssembler");
    lua_pushnumber(L, c);
    lua_pushnumber(L, j);
    lua_method("get_shapeg_vec", 3, 0);
}


\end{verbatim}
\subsection{Setting and getting {\tt U, V, A, F}}

There are a lot of variants of setting and getting the system state
vectors, simply because they're all potentially complex-valued.
These functions all copy to/from reduced vectors, with part of the
data merged in from the boundary value arrays.

\subsection{Lua support methods}

These routines support access to the Lua side of the mesh object.
The {\tt lua\_getfield}, {\tt lua\_setfield}, and {\tt lua\_method}
methods let us access fields and methods associated with the Lua
object that represents the mesh from tolua.

\begin{verbatim}


void Mesh::lua_getfield(const char* name)
{
    tolua_pushusertype(L, this, "Mesh");
    lua_pushstring(L, name);
    lua_gettable(L, -2);
    lua_remove(L, -2);
}


void Mesh::lua_setfield(const char* name)
{
    tolua_pushusertype(L, this, "Mesh");
    lua_insert(L,-2);
    lua_pushstring(L, name);
    lua_insert(L,-2);
    lua_settable(L,-3);
    lua_pop(L,1);
}


int Mesh::lua_method(const char* name, int nargin, int nargout)
{
    lua_getfield(name);
    if (!lua_isfunction(L,-1)) {
        lua_pop(L,nargin+1);
        return -1;
    }
    lua_insert(L,-1-nargin);
    tolua_pushusertype(L, this, "Mesh");
    lua_insert(L,-1-nargin);
    return lua_pcall2(L, nargin+1, nargout);
}
\end{verbatim}
